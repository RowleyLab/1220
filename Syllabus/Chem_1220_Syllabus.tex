\documentclass[12pt, letterpaper]{article}
\usepackage{SyllabusStyle}

\begin{document}
\begin{center}
{\Large \textsc{Principles of Chemistry II}}

CHEM 1220
\end{center}
\begin{center}
{\large Spring 2022}
\end{center}
\begin{center}
	\rule{0.99\textwidth}{0.4pt}
	\begin{tabular}{llcll}
		\textbf{Instructor:} & Matthew Rowley & & \textbf{Office Hours:} & Daily 10:00 am -- 11:00 am \\
		\textbf{Telephone:} & (435) 586-7875 & & & \\
		\textbf{Email:} & matthewrowley$1$@suu.edu  & & \textbf{Office:} & SC-220\\
		\multicolumn{5}{c}{Please include the course number in the subject line of all correspondence.} 
	\end{tabular}
	\rule{0.99\textwidth}{0.4pt}
\end{center}

\section*{Course Description} 
This is an introductory chemistry course designed for students in engineering, physical science, pre-medical, pre-dental, pre-pharmacy, or pre-veterinary medicine. It is for all students who need more than one year of chemistry, and presents the foundational principles on which all other chemistry courses build.

\paragraph{General Education:}
This course is required for beginning Chemistry majors requiring more than one year of chemistry. It can also be taken by other students to fulfill the General Education (GE) requirements for ``Knowledge Area:Physical Science (P).'' The principles of chemistry underlie much of modern society. Students will explore the basic principles of chemistry, allowing them to better understand the world around them at a basic level. This will prepare students to be informed citizens of the modern world in topics such as medicine, materials, energy and pollution.

\paragraph{Prerequisites:}
A proficient understanding of algebra will be required for this course. Either MATH 1050 (College Algebra), or a background of algebra from high school will suffice.

\paragraph{Concurrent requisite:}
CHEM 1225 -- Principles of Chemistry Lab II

\paragraph{Required Course Materials:} ~

Access to \emph{Achieve} online homework and Interactive General Chemistry digital textbook (details below)

\paragraph{Recommended Supplementary Materials:} ~

\emph{Preparing for Your ACS Examination in General Chemistry: The Official Guide} by Eubanks and Eubanks (ISBN: 978-0-970-80420-4)

\paragraph{Student Learning Outcomes:}
\begin{description}
  \item[Knowledge of the Physical and Natural World] Students will recall, interpret, compare, explain, and apply chemistry terminology and theory
  \item[Quantitative Literacy] Students will use chemical equations, graphs, and tables to interpret and communicate chemical information.
  \item[Inquiry and Analysis] Students will solve complex chemical problems.
  \item[Critical Thinking] Students will make decisions based on conceptualizing, applying, and analyzing information from different sources.
\end{description}

\section*{Tentative Schedule}
Class will meet on Mondays, Wednesdays, Thursdays, and Fridays from 12:00-12:50 in room SC-214

\noindent For the best lecture experience, read the indicated textbook chapter \emph{before} viewing each lecture

\begin{tabular}{rcccc}
	& Date && Topic & Chapter\\
	\midrule
	Week 1 & \multirow{2}{*}{M, Jan. 10}& & Intermolecular Forces and Liquid Properties & 12.1-12.2\\
	& & & Phase Changes and Heating Curves & 12.3\\
	& W, Jan. 12&& Vapor Pressure and Phase Diagrams & 12.4-12.5\\
	& R, Jan. 13&& Classifying Solids and Unit Cells & 12.6-12.7\\
	& F, Jan. 14&& Solvation and Saturation & 13.1-13.2\\
	\midrule
	Week 2 & M, Jan. 17& \multicolumn{3}{l}{\textbf{Martin Luther King Day - No Class!}}\\
	& W, Jan. 19&& Concentration Units & 13.3\\
	& R, Jan. 20&& Colligative Properties & 13.4-13.5\\
	& F, Jan. 21& \multicolumn{3}{l}{\textbf{Catch-up/Review Day} - Midterm Exam 1 (Ch. 12--13)}\\
	\midrule
	Week 3 & M, Jan. 24&& Rates and Rate Laws & 14.1-14.2\\
	& W, Jan. 26&& Integrated Rate Laws & 14.3\\
	& \multirow{2}{*}{R, Jan. 27}& & Temperature and Activation Energy & 14.4\\
	& & & Reaction Mechanisms and Catalysis & 14.5-14.6\\
	& F, Jan. 28&& Equilibrium Constants & 15.1-15.2\\
\end{tabular}

\begin{tabular}{rcccc}
    & Date && Topic & Chapter\\	
	\midrule
	Week 4 & M, Jan. 31&& Equilibrium Expressions and Q & 15.3-15.4\\
	& W, Feb. 2&& ICE Tables & 15.5\\
	& R, Feb. 3&& Le Ch\^atelier's Principle & 15.6\\
	& F, Feb. 4&& Acid and Base Reactions & 16.1-16.2\\
	\midrule
	Week 5 & M, Feb. 7&& Autoionization and pH & 16.3-16.4\\
	& W, Feb. 9&& Weak Acids and Bases & 16.5\\
	& R, Feb. 10&& Polyprotic Acids and Salts & 16.6-16.7\\
	& F, Feb. 11&& Acid Strength and Lewis Acids & 16.8-16.9\\
	\midrule
	Week 6 & M, Feb. 14& \multicolumn{3}{l}{\textbf{Catch-up/Review Day} - Midterm Exam 2 (Ch. 14--16)}\\
	& W, Feb. 16&& Buffers and the H-H Equation & 17.1-17.2\\
	& R, Feb. 17&& Strong Acid/Base Titrations & 17.3\\
	& F, Feb. 18&& Weak Acid/Base Titrations & 17.4-17.5\\
	\midrule
	Week 7 & M, Feb. 21& \multicolumn{3}{l}{\textbf{President's Day - No Class!}}\\
	& W, Feb. 23&& Solubility & 17.6-17.7\\
	& R, Feb. 24&& Precipitation and Q & 17.8\\
	& F, Feb. 25&& Metal Ions and Complexation & 17.9-17.10\\
	\midrule
	Week 8 & M, Feb. 28& \multicolumn{3}{l}{\textbf{Spring Break - No Class!}}\\
	& W, Mar. 2& \multicolumn{3}{l}{\textbf{Spring Break - No Class!}}\\
	& R, Mar. 3& \multicolumn{3}{l}{\textbf{Spring Break - No Class!}}\\
	& F, Mar. 4& \multicolumn{3}{l}{\textbf{Spring Break - No Class!}}\\
	\midrule
	Week 9 & M, Mar. 7&& Entropy and Spontaneity & 18.1\\
	& W, Mar. 9&& Entropy Changes and Temperature & 18.2-18.3\\
	& R, Mar. 10&& Gibbs Energy and Temperature & 18.4-18.5\\
	& F, Mar. 11&& Gibbs Energy and Equilibrium & 18.6\\
\end{tabular}

\begin{tabular}{rcccc}
	& Date && Topic & Chapter\\	
	\midrule
	Week 10 & M, Mar. 14&& Redox Reactions & 19.1-19.3\\
	& W, Mar. 16&& Voltaic Cells & 19.4-19.5\\
	& R, Mar. 17&& Free Energy and Cell Potential & 19.6\\
	& F, Mar. 18&& Nernst Equation and Applications & 19.7\\
	\midrule
	Week 11 & M, Mar. 21&& Electrochemical Cell Applications & 19.8-19.9\\
	& W, Mar. 23& \multicolumn{3}{l}{\textbf{Catch-up/Review Day} - Midterm Exam 3 (Ch. 17--19)}\\
	& R, Mar. 24&& Radioactivity & 20.1-20.2\\
	& F, Mar. 25&& Half-Life and Radiometric Dating & 20.3-20.4\\
	\midrule
	Week 12 & M, Mar. 28&& Fission and Fusion & 20.5\\
	& W, Mar. 30& \multicolumn{3}{l}{\textbf{Festival of Excellence - No Class!}}\\
	& R, Mar. 31&& Energy and Nuclear Reactions & 20.6-20.7\\
	& F, Apr. 1&& Hydrocarbons & 21.1-21.2\\
	\midrule
	Week 13 & M, Apr. 4&& Isomers & 21.3\\
	& W, Apr. 6&& Classes of Organic Compounds & 21.4-21.5\\
	& R, Apr. 7&& Polymers & 21.6\\
	& F, Apr. 8&& Transition Metals & 22.1-22.2\\
	\midrule
	Week 14 & M, Apr. 11&& Coordination Compounds & 22.3\\
	& W, Apr. 13&& Nomenclature and Isomerism & 22.4-22.5\\
	& R, Apr. 14&& Crystal Field Theory and Spectroscopy & 22.6-22.7\\
	& F, Apr. 15& \multicolumn{3}{l}{\textbf{Catch-up/Review Day} - Midterm Exam 4 (Ch. 20--22)}\\
	\midrule
	Week 15 & M, Apr. 18&& Carbohydrates & 23.1-23.2\\
	& W, Apr. 20&& Lipids & 23.3\\
	& R, Apr. 21&& Amino Acids & 23.4\\
	& F, Apr. 22&& Nucleic Acids & 23.5\\
	\midrule
	\midrule
	Finals Week & T, Apr. 26& \multicolumn{3}{l}{\textbf{Final Exam} -- 11:00-12:50 ~~ \emph{Bring a Scantron and a Pencil!}}\\
\end{tabular}

\section*{Course Requirements}
Grades will be based on the following items:
\begin{description}
  \item[4 Midterm Exams] 40\%
  \item[Final Exam] 15\%
  \item[Quizzes] 15\%
  \item[Online Homework] 15\%
  \item[Daily Textbook Homework] 15\%
\end{description}
Final Grades will be assigned according to the following grade scale:

\begin{tabular}{cl|c|cl}
	Percentage & Grade &  & Percentage & Grade \\ \midrule
	100--93.0 & A     &  &  77.0--73.0 & C     \\
	93.0--90.0 & A-    &  &  73.0--70.0 & C-    \\
	90.0--87.0 & B+    &  &  70.0--67.0 & D+    \\
	87.0--83.0 & B     &  &  67.0--63.0 & D     \\
	83.0--80.0 & B-    &  &  63.0--60.0 & D-    \\
	80.0--77.0 & C+    &  &     < 60.0 & F
\end{tabular}

\paragraph{Midterm Exams:}
There are four midterm exams administered in the testing center through Canvas. Each exam is to be completed in a two-hour session during the indicated week unless prior arrangements have been made. A sparse resource sheet will be provided for each exam, but students should prepare by memorizing appropriate formulas, tables, etc.

\paragraph{Final Exam:}
The final exam is a comprehensive and nationally normalized exam prepared by the American Chemical Society.

\paragraph{Quizzes:}
Quizzes will be given either in class or as a ``take-home'' quiz. The purpose of these quizzes is to provide practice for the exams and to encourage punctual attendance.

\paragraph{Achieve Online Homework:}
The Achieve online homework assignments are organized by chapter and are of substantial length. I recommend completing the assignments in multiple sessions (Achieve saves your work), as we cover new material each day. You can find a link to sign up for Achieve in our Canvas course.

\paragraph{Daily Textbook Homework}
Most days will end with an assignment of a few problems from our textbook. These problems mostly have solutions in the back of the book, so you are encouraged to check your work and try to correct your answers if wrong. The assignments are graded on participation only, and are intended to encourage daily engagement with the material.

\paragraph{Attendance Policy:}
Students are expected to attend class. If you must miss class, contact the instructor

\paragraph{Late Work Policy:}
Textbook homework and take-home quizzes will be due on a day when class is regularly scheduled. All work is to be turned in at the \emph{beginning} of the class period, and late work will not be accepted. Please note that online homework has due-dates at 11:55 pm, and will not be available after that time

\paragraph{Make-up Work Policy:}
In general, there will be no opportunity to make up missed work, including in-class quizzes. If you must miss class, please do any assigned work in advance, and arrange to turn it in early

\section*{Miscellany}

\paragraph{Scientific Calculator:}
There are many different ways to calculate figures during homework. It is tempting to rely on Online resources such as \href{http://www.wolframalpha.com}{http://www.wolframalpha.com}, or to simply use a calculator application on a smart phone. During exams, however, any devices capable of connecting to the Internet will \emph{not} be allowed. You will instead need a scientific calculator capable of performing exponentiation and logarithms for the exams. Using this calculator exclusively while doing homework will ensure that you are familiar with it for use during exams.

\paragraph{Academic Integrity:}
Scholastic dishonesty will not be tolerated and will be prosecuted to the fullest extent. You are expected to have read and understood the current issue of the \href{https://help.suu.edu/handbook}{Student Handbook} (published by Student Services) regarding student responsibilities and rights, and for the intellectual property policy, information about procedures, and what constitutes acceptable behavior. From University policy 6.33: ``The University defines plagiarism as intentionally or carelessly presenting the work of another as one’s own. It includes submitting an assignment purporting to be the student’s original work which has wholly or in part been created by another person, or cutting and pasting of source material\ldots''

\paragraph{ADA Policy:}
Students with medical, psychological, learning, or other disabilities desiring academic adjustments, accommodations, or auxiliary aids will need to contact the Southern Utah University Coordinator of Services for Students with Disabilities (SSD), in Room 206F of the Sharwan Smith Center or phone (435) 865-8022. SSD determines eligibility for and authorizes the provision of services.

\paragraph{Emergency Management Statement:}
In case of emergency, the university's Emergency Notification System (ENS) will be activated. Students are encouraged to maintain updated contact information using the link on the homepage of the \emph{mySUU} portal. In addition, students are encouraged to familiarize themselves with the Emergency Response Protocols posted in each classroom. Detailed information about the university's emergency management plan can be found at: \href{http://www.suu.edu/emergency}{http://www.suu.edu/emergency}

\paragraph{HEOA Compliance Statement:}
The sharing of copyrighted material through peer-to- peer (P2P) file sharing, except as provided under U.S. copyright law, is prohibited by law. Detailed information can be found at: \href{https://help.suu.edu/article/1097/p2p-and-copyright-infringement}{https://help.suu.edu/article/1097/p2p-and-copyright-infringement}

\paragraph{LINK Statement:}
SUU faculty and staff care about the success of our students. In addition to your professor, numerous services are available to assist you with the achievement of your educational goals. SUU's LINK system may be used by faculty to notify you and/or your advisors of their concern for your progress and provide references to campus services as appropriate.

\paragraph{SUUSA Statement:}
As a student at SUU, you have representation from the SUU Student Association (SUUSA) which advocates for student interests and helps work as a liaison between the students and the university administration. You can submit My SUU Voice feedback by going here: \href{https://www.suu.edu/suusa/voice}{https://www.suu.edu/suusa/voice} Likewise, you can learn more about SUUSA's Executive Council here (\href{https://www.suu.edu/suusa/executive-council/}{https://www.suu.edu/suusa/executive-council/}) and about indivdual SUUSA's Student Sentors here (\href{https://www.suu.edu/suusa/senate/}{https://www.suu.edu/suusa/senate/})

\paragraph{University Policies and Recommendations Regarding COVID-19:}
Southern Utah University has compiled a collection of information, policies, and recommendations related to COVID-19 at \href{https://www.suu.edu/coronavirus/}{suu.edu/coronavirus/}

\noindent I dearly want this semester to go smoothly vis-\`a-vis COVID-19, and I assume you all do as well. Toward that end, I encourage you all to exercise all reasonable precaution to prevent the spread of the coronavirus. This includes using the testing and self-reporting resources at the link above.

\noindent It may interest some of you to know that, for my part, I have been vaccinated with two doses of the Moderna vaccine, and more recently received a Pfizer booster. I will wear a mask on campus when appropriate. I will \emph{not} be wearing a mask as I lecture, since clear communication is my primary goal in the classroom.

\paragraph{Disclaimer:}
Information contained in this syllabus, other than the grading, late assignments, make up work and attendance policies, may be subject to change as deemed appropriate by the instructor.
\end{document}