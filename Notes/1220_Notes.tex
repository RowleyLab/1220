\documentclass[12pt, openany, letterpaper]{memoir}
\usepackage{NotesStyle}
%\renewcommand\thesection{\thechapter\Alph{section}}
%\renewcommand\thesubsection{\thesection.\Numeral{subsection}}

\begin{document}
\title{CHEM 1220 Lecture Notes\\ OpenStax Chemistry 2e}
\author{Matthew Rowley}
\date{\today}
\mainmatter
\maketitle
\chapter*{Course Administrative Details}
\begin{itemize}
	\item My office hours
	\item Intro to my research
	\item Introductory Quiz
	\item Grading details
	      \begin{itemize}
		      \item Exams - 40, Final - 15, Online Homework - 15, Book Homework - 15, Quizzes - 15
		      \item Online homework
		      \item Frequent quizzes
	      \end{itemize}
	\item Importance of reading and learning on your own
	\item Learning resources
	      \begin{itemize}
		      \item My Office Hours
		      \item Tutoring services - \href{https://www.suu.edu/academicsuccess/tutoring/}{https://www.suu.edu/academicsuccess/tutoring/}
	      \end{itemize}
	\item Show how to access Canvas
	      \begin{itemize}
		      \item Calendar, Grades, Modules, etc.
		      \item Quizzes
		      \item Textbook
	      \end{itemize}
	\item Introduction to chemistry
	      \begin{itemize}
		      \item Ruby fluorescence
		      \item Levomethamphetamine
		      \item Rubber band elasticity
		      \item Structure of the periodic table
		      \item Salt on ice and purifying hydrogen peroxide
	      \end{itemize}
\end{itemize}

\setcounter{chapter}{-1}
\chapter{1210 Review}

There is a whole semester of material from 1210, and these are only the topics which are \emph{most} important for success in 1220

\begin{itemize}
	\item Composition of atoms and ions (protons, neutrons and electrons)
	\item Chemical formulas and names
	\begin{itemize}
		\item Formulas and molar masses
		\item Polyatomic ion names
		\item Naming ionic compounds
		\item Naming binary molecular compounds
		\item Naming acids
	\end{itemize}
	\item Balancing molecular equations
	\item Solubility rules
	\item Fundamentals of acid/base chemistry
	\item Measurements in chemistry
	\begin{itemize}
		\item Converting from measurements to moles and back
		\item Stoichiometry and predicting amounts
		\item Limiting reactants
	\end{itemize}
	\item Enthalpy of reaction and heat equations
	\item Lewis structures
\end{itemize}
\paragraph*{CHEM 1210 Review Quiz}

\setcounter{chapter}{9}
\chapter{Liquids and Solids}

\section{Intermolecular Forces}
\begin{itemize}
  \item Many physical properties of solids, liquids, and gases can be explained by the strength of attractive forces between particles (Figure 10.5)
  \item Phase changes happen due to the interplay between kinetic energy and intermolecular forces (Figure 10.2)
  \item Pressure can also play a role in phase changes, as discussed later
  \item These \emph{intermolecular forces} come in different varieties
  \begin{itemize}
    \item Dispersion Forces Non-polar molecules, impacted by polarizability, molecular weight, and surface area
    \begin{itemize}
      \item Dominant in non-polar molecules
      \item Created by induced dipoles (Figure 10.6)
      \item Impacted by polarizability (Table 10.1)
      \item Impacted by molecular weight (hydrocarbons from methane to wax)
      \item Impacted by molecule shape (Figure 10.7 compares the boiling points of pentane isomers)
    \end{itemize}
    \item Dipole-Dipole Forces
    \begin{itemize}
      \item Dominant in polar molecules
      \item Results from attraction between permanent dipoles (Figure 10.9)
    \end{itemize}
  \item Hydrogen Bonding
    \begin{itemize}
      \item Dominant only in molecules capable of hydrogen bonding
      \item Must contain a hydrogen-donor atom (H attached to N, O, or F)
      \item Must contain a hydrogen-acceptor atom (lone pair of electrons attached to N, O, or F)
      \item Hydrogen bonds are more than just particularly strong dipole-dipole forces. They have strong directionality according to VSEPR
      \item Figures 10.10, 10.14, and other figures on the Internet show water, DNA, and proteins all organized by hydrogen bonds
      \item Figures 10.11 and 10.12 illustrate how much hydrogen bonds exceed dipole-dipole forces in strength
    \end{itemize}
  \end{itemize}
\end{itemize}

\section{Properties of Liquids}
\begin{itemize}
  \item Viscosity is a fluid's resistance to flow
  \begin{itemize}
    \item We intuitively know that both water and honey flow\ldots but at very different rates
    \item Viscosity is proportional to the strength of intermolecular forces (high IF = high viscosity)
    \item As temperature increases, kinetic energy is able to overcome intermolecular forces and viscosity decreases
    \item Table 10.2 gives the viscosities of some common substances (note the unusual units!)
  \end{itemize}
  \item Surface tension is a force which minimizes a fluid's surface area
  \begin{itemize}
    \item Cohesive vs. adhesive forces
    \item Bulk molecules have lower energy than surface molecules due to being \emph{surrounded} by cohesive forces (Figure 10.16)
    \item Figure 10.17 illustrates a waterbug supporting itself on water surface tension
    \item Surface tension is often in conflict with gravity and other forces, making most liquids rounded but not perfect spheres
    \item Surface tension is proportional to intermolecular forces (Table 10.3)
    \item Surface tension can be strongly affected by addition of certain solutes, called surfactants
  \end{itemize}
  \item Capillary action is a force between a fluid and narrow channels or capillaries of solid materials
  \begin{itemize}
    \item Due to adhesive forces with the solid, liquids will be drawn up (or, less often, pushed down) a capillary
    \item Figure 10.19 shows how paper towels are made to maximize capillary action, so they soak up water-based spills
    \item The top of the liquid (called the meniscus) will curve differently depending on the reletive strength of cohesive and adhesive forces (Figure 10.18)
    \item Figure 10.20 shows capillary action in a variety of situations, including capillary repulsion
    \item Remember that when measuring volumes, convention is to read the \emph{bottom} of the meniscus regardless of how it curves
    \item Don't worry about the formula given here
  \end{itemize}
\end{itemize}
\paragraph*{Quiz 10.1 - Intermolecular Forces and Liquid Properties}
\paragraph*{Homework 10.1}
\begin{itemize}
  \item 10.11: Predicting trends in boiling points
  \item 10.21: Identifying intermolecular forces
  \item 10.25: Affect of temperature on viscosity
\end{itemize}

\section{Phase Transitions}
\begin{itemize}
  \item Vaporization and condensation are the transitions between liquid and gas phases
  \begin{itemize}
    \item The enthalpy of vaporization $\left(\Delta H_{vap}\right)$ is the energy required to transition from liquid to gas phase
    \item Enthalpy of condensation is the opposite $\Delta H_{con} = - \Delta H_{vap}$
    \item In a closed volume, these processes will reach a \emph{dynamic equilibrium}
    \item The partial pressure of the liquid at this equilibrium state is called its \emph{vapor pressure} (Figure 10.22)
    \item Higher intermolecular forces lead to lower vapor pressures
    \item Higher temperatures increase the vapor pressure due to increased kinetic energy (Figure 10.23)
  \end{itemize}
  \item Boiling points
  \begin{itemize}
    \item Figure 10.24 shows vapor pressure curves and the normal boiling points of several liquids
    \item Boiling points generally depend on the pressure (pressure cookers, boiling water to freezing, etc.)
    \item The Clausius-Clapeyron equation defines these curves (Note the rearrangments I've made)
      \\ $P=Ae^{\nicefrac{-\Delta H_{vap}}{RT}}$ \hspace{2em} 
      $\ln P = -\dfrac{\Delta H_{vap}}{RT}+\ln A$ \hspace{2em}
      $\ln\left(\dfrac{P_2}{P_1}\right)=-\dfrac{\Delta H_{vap}}{R}\left(\dfrac{1}{T_2}-\dfrac{1}{T_1}\right)$ 
  \end{itemize}
  \item Fusion (melting), freezing, sublimation, and deposition all have their enthalpies and transition temperatures
  \item These enthalpies are state functions, such that $\Delta H_{sub} = \Delta H_{fus}+\Delta H_{vap}$ (Figure 10.28)
  \item Heating and Cooling curves
  \begin{itemize}
    \item When heat is added to a system, it will either cause a phase change, or a change in temperature
    \item For phase changes, $q=n\Delta H_{change}$
    \item For temperature changes, $q=mc\Delta T$, where $c$ is the specific heat for that substance and phase
    \item Sometimes $\Delta H_{change}$ is given as a -per gram value, and sometimes $c$ is given as a -per mole value, but usually not :(
    \item Figure 10.29 shows a typical heating curve (Work example 10.10 in the text)
  \end{itemize}
\end{itemize}
\paragraph*{Quiz 10.2 - Heating Curves}
\paragraph*{Homework 10.2}
\begin{itemize}
  \item 10.31: Temperature during a phase transition
  \item 10.39: Definition of normal boiling point
  \item 10.51: Heating curve problem
\end{itemize}

\section{Phase Diagrams}
\begin{itemize}
  \item The stable phase at different temperatures and pressures is best illustrated with a phase diagram (Figures 10.30, 10.31)
  \item We can tell at a glance what transitions might occur as we increase or decrease either the temperature or pressure
  \item Note that at some pressures, sublimation may occur instead of fusion
  \item The triple point is a unique point where liquid, solid, and gas can all exist at equilibrium (contrast with a glass of icy water on a humid day)
  \item The critical point is where the distinction between liquid and solid phases disappears
  \item Figure 10.34 shows the phase diagram of \ch{CO2}
  \item Supercritical fluids exhibit some interesting properties, and are often great solvents (Nile Blue Youtube video)
  \item Critical points vary widely depending on the intermolecular forces, and other factors (Table in text)
\end{itemize}
\paragraph*{Quiz 10.3 - Phase Diagrams}
\paragraph*{Homework 10.3}
\begin{itemize}
  \item 10.55: Trajectories on a phase diagram
  \item 10.57: Determining state on a phase diagram
  \item 10.63: Identifying phases on a blank phase diagram
\end{itemize}

\section{The Solid State of Matter}
\begin{itemize}
  \item Solids can be divided into \emph{crystalline} and \emph{amorphous} based on their structure at atomic scales
  \item Figure 10.37 shows the difference generally, Figure 10.38 shows crystalline and amorphous \ch{SiO2}
  \item Amorphous solids will not exhibit a sharp fusion transition temperature, but will instead grow soft and maleable over a temperature range
  \item Crystalline solids are diverse but always show long-range repeating order in their structure
  \begin{itemize}
    \item Ionic solids (Figure 10.39) have high melting points, cleave along planes, and conduct electricity only in the liquid phase
    \item Metallic solids (Figure 10.40) have mostly high melting points, are maleable and ductile, and conduct electricty and heat well
    \item Covalent network solids (Figure 10.41) have very high melting points and are electrical insulators
    \item Molecular solids (Figure 10.42) Have low to very low melting points and are electrical insulators
    \item Crystalline solid properties are summarized in Table 10.4
  \end{itemize}
  \item Even crystalline solids do not have perfect structure. Various types of defects are illustrated in Figure 10.45
\end{itemize}
\paragraph*{Quiz 10.4 - Classifying Solids}
\paragraph*{Homework 10.4}
\begin{itemize}
  \item 10.69: Classify solids by formulas
  \item 10.71: Classify solids by properties
\end{itemize}

\section{Lattice Structures in Crystalline Solids}
\begin{itemize}
  \item The structure of a crystalline solid is represented by a \emph{unit cell}, the smallest repeatable unit of the structure
  \item Sometimes this microscopic structure is evidently manifested on macroscopic scales, but sometimes it isn't
  \item Unit cells are defined by lattice points that often lie at the center of certain atoms, and the cell edges often cut atoms in half, quarter, etc.
  \item Unit cells of metals
  \begin{itemize}
    \item For metals, we should keep track of the quantity of atoms in a unit cell, the coordination number, and the relationship between the atomic radius and unit cell edge length
    \item Simple cubic (Figure 10.49) 1 atom, Coordination=6, $l=2r$
    \item Body-centered cubic (Figure 10.51) 2 atoms, Coordination=8, $l=\frac{4}{\sqrt{3}}r$
    \item Face-centered cubic (Figure 10.52) 4 atoms, Coordination=12, $l=\sqrt{8}r$
    \item Figure 10.54 shows hexagonal closest packed and cubic closest packed structures
    \item Find the radius of a gold atom, which has fcc structure and a density of $19.283\nicefrac{g}{cm^3}$ ($136pm$)
    \item Find the density of polonium, which has sc structure and an atomic radius of $140pm$ ($9.20\nicefrac{g}{cm^3}$)
    \item Figure 10.56 shows many non-cubic structures which are common as well
  \end{itemize}
  \item Unit cells of ionic compounds
  \begin{itemize}
    \item Anions are generally larger than cations, so ionic lattice points are generally the centers of anions
    \item Cations occupy holes in the anionic lattice (Figures 10.57 and 10.58)
    \item Unit cells of ionic structures share names with the metallic cells but look different because of the cations
    \item Simple cubic (Figure 10.59)
    \item Face-centered cubic (rock salt structure) (Figure 10.60)
    \item Zinc blende (Figure 10.61)
    \item Find the ionic bond length for \ch{NaCl} which has rock salt structure and denisty of $2.17\nicefrac{g}{cm^3}$ ($l=564pm$)
  \end{itemize}
  \item Crystal structure is determined through X-ray crystallography
  \begin{itemize}
    \item X-rays reflected off a crystal surface can combine destructively or constructively to produce an interference pattern (Figure 10.63)
    \item The X-rays will take different pathlengths depending on the angle of the X-ray beam and the crystal lattice constant (Figure 1.64)
    \item An experimental setup and actual diffractogram are shown in Figures 10.65 and 10.66
    \item We have a powerful X-ray instrument here at SUU
  \end{itemize}
\end{itemize}
\paragraph*{Quiz 10.5 - Unit Cells}
\paragraph*{Homework 10.5}
\begin{itemize}
  \item 10.77: Coordination number
  \item 10.81: Density from lattice constant
  \item 10.85: Packing efficiency and density
\end{itemize}

\chapter{Solutions and Colloids}

\section{The Dissolution Process}
\begin{itemize}
  \item Some vocabulary: \emph{Solution}, \emph{Solvent}, \emph{Solute}, and \emph{solvation}
  \item Table 11.1 shows many different types of solutions, with different phases of solvent and solute
  \item Molecular compounds dissolve to form one solute:

    \ch{C6H12O6(s) -> C6H12O6(aq)}
  \item Ionic compounds will dissolve into individual ions:

    \ch{Na2SO3(s) -> Na2SO3(aq) -> 2 Na^+(aq) + SO3^{2-}(aq)}
  \item Dissolving soluble compounds is a thermodynamically \emph{spontaneous} process
  \begin{itemize}
    \item Spontaneity is covered in more detail in chapter 16
    \item Solvation mixes solvent and solute, increasing the system \emph{entropy} (Figure 11.3)
    \item Solvation can be either \emph{exothermic} (favors spontaneity) or \emph{endothermic} (hampers spontaneity) depending on the strength of solvent-solvent, solute-solute, and solvent-solute intermolecular forces (Figure 11.4)
    \item Demonstration, dissolving \ch{NaOH(s)} and \ch{NH4NO3(s)} in water (Don't overdo the \ch{NaOH}!)
    \item When solvation has $\Delta H \approx 0$, the result is an \emph{ideal solution}, whose properties best match simple laws
  \end{itemize}
\end{itemize}

\section{Electrolytes}
\begin{itemize}
  \item Electrolytes will yield ions when dissolved in water, yielding a solution which conducts electricity (Figure 11.6)
  \begin{description}
    \item[Non-electrolytes:] Do not yield ions at all when dissolved (Most molecular compounds)
    \item[Strong electrolytes:] Produce a large (stoichiometric) amount of ions when dissolved (Soluble ionic compounds and \emph{strong} acids/bases)
    \item[Weak electrolytes:] Produce a smaller amount of ions when dissolved (\emph{weak} acids/bases)
  \end{description}
  \item Ionic electrolytes produce ions by directly \emph{dissociating} into their cations and anions (Figure 11.7)
  \item Molecular electrolytes produce ions by reacting with the solvent or other molecules

    \ch{NH3(aq) + H2O(l) <=> NH4^+(aq) + OH^-(aq)}
\end{itemize}

\section{Solubilty}
\begin{itemize}
  \item Table 4.1 gave rules to predict if an ionic compound is soluble or insoluble, but in reality solubilities lie on a spectrum
  \item In Chapter 15, we will explore solubility with mathematic rigor. For now, we will focus on trends and factors affecting solubility
  \item Solubility is a type of reaction governed by \emph{equilibrium}
  \begin{description}
    \item[Unsaturated] solutions have not yet reached their limit of how much solute they can dissolve
    \item[Saturated] solutions have met their solubility limit and are in equilibrium. You can recognize a saturated solution by the presence of undissolved solute in contact with the solution
    \item[Supersaturated] solutions have exceeded their solubility limit. This situation is only \emph{metastable} and usually contrived by quick changes in temperature or volume (``Jeremy Krug'' Youtube video of supersaturated \ch{NaCH3CO2})
  \end{description}
  \item Solutions of gases in liquids
  \begin{itemize}
    \item Gas-in-liquid solvation is always exothermic and solubility depends primarily on solvent-solute interactions
    \item Solubility is decreases as temperature rises (Figures 11.8 and 11.9)
    \item Solubility also depends on the gas partial pressure, according to Henry's law. Figure 11.8 gives $k_H$, and Figure 11.10 illustrates how to use Henry's law to supersaturate a solution (carbonation!)

      $C_{gas}=k_HP_{gas}$
  \end{itemize}
  \item Solutions of liquids in liquids
  \begin{itemize}
    \item Miscible liquids are infinitely soluble in each other (mix in any ratio)
    \item Immiscible liquids have very low solubility in each other, and separate to form layers. Oil and water (Figure 11.14) are a classic example of immiscible liquids and illustrate the axiom that ``like dissolves like'' because their intermolecular forces are so different
    \item Partially miscible liquids will form two layers when mixed, but each layer contains significant amounts of the other solute liquid
  \end{itemize}
  \item Solutions of solids in liquids
  \begin{itemize}
    \item Figure 11.6 shows the temperature dependance of solubility for several solids
    \item \emph{Exothermic} $\Delta H_{solv}$ leads to lower solubility at higher temperatures
    \item \emph{Endothermic} $\Delta H_{solv}$ leads to higher solubility at higher temperatures
  \end{itemize}
\end{itemize}
\paragraph*{Quiz 11.1 - The Solvation Process}
\paragraph*{Homework 11.1}
\begin{itemize}
  \item 11.3: Energetics of solvation
  \item 11.9: Rule of like dissolves like
  \item 11.13: Classifying electrolytes
  \item 11.23: Henry's law
\end{itemize}

\section{Colligative Properties}
\begin{itemize}
  \item Colligative properties of solutions depend on the \emph{amount} of solute present, regardless of the chemical identity of the solutes
  \item Some colligative properties depend on less common units of concentration
  \begin{itemize}
    \item Recall molarity from chapter 3
    \item \emph{Mass \%} is $\dfrac{m_{solute}}{m_{total}}\times 100\%$
    \item \emph{Mole fraction} is $\chi_A=\dfrac{n_A}{n_{total}}$
    \item \emph{Molality} is $m=\dfrac{moles_{solute}}{kg_{solvent}}$
    \item Practice interconverting between these units: $\chi_{\ch{C6H12O6}}=0.25$ in aqueous solution
    \item For electrolytes, we will also need the van't Hoff factor, $i = $ \# of particles produce on solvation
  \end{itemize}
\end{itemize}

\paragraph*{Quiz 11.2 - Concentrations}
\paragraph*{Homework 11.2}
\begin{itemize}
  \item 11.19: \% by mass and solubility
  \item 11.31: Mole fraction
  \item 11.39: Molality
\end{itemize}

\paragraph*{Back to Section 11.4  Colligative Properties}
\begin{itemize}
  \item Vapor pressure lowering
  \begin{itemize}
    \item Figure 11.8 illustrates why solutes lower the vapor pressure of the solvent
    \item Rault's law: $P_A=\chi_AP^\star_A$
    \item If the solute is a liquid, we can apply Rault's law to the solute as well $P_{total}=\chi_AP^\star_A+\chi_BP^\star_B$
    \item This gives a different composition of the gas phase from the liquid phase, allowing for purification through distillation (Figures 11.19 and 11.20)
  \end{itemize}
  \item Changes in phase tranistion temperatures
    \begin{itemize}
      \item Boiling point elevation is a consequence of vapor pressure lowering
      \item Freezing point depression follows a similar forumula 
      \item $\Delta T_{f/b}=iK_{f/b}m$
      \item Table 11.2 gives $T_b$, $K_b$, $T_f$, and $K_f$ for several substances
      \item These effects manifest on a phase diagram like in Figure 11.23
    \end{itemize}
  \item Osmotic pressure
  \begin{itemize}
    \item Figure 11.24 shows how water will flow across a selectively permeable membrane via osmosis
    \item Figure 11.25 shows how and applied pressure can reverse this process and purify water
    \item $\pi=\dfrac{inRT}{V} = iMRT$
    \item Figure 11.27 shows how blood salinity can impact the health of red blood cells (isotonic, hypertonic, hypotonic, hemolysis, crenation)
  \end{itemize}
  \item Measuring colligative properties can give the molar mass of an unknown, as long as we know the van't Hoff factor
\end{itemize}
\paragraph*{Quiz 11.3 - Colligative Properties}
\paragraph*{Homework 11.3}
\begin{itemize}
  \item 11.45: Freezing point depression
  \item 11.61: Osmotic pressure
  \item 11.65: Vapor pressures of mixtures
\end{itemize}

\section{Colloids}
\begin{itemize}
  \item Colloids occupy the blurry boundry region between homogeneous and heterogeneous mixtures (Figure 11.29)
  \item Colloids can be identified by several properties:
  \begin{itemize}
    \item Particle size is on a range of tens- to hundreds- of nanometers
    \item Particles will not settle out on their own under the influence of gravity
    \item Particles will scatter light, called the Tyndall Effect (Figure 11.30)
  \end{itemize}
  \item Table 11.4 gives some examples of colloids in various phases
  \item Emulsifying agents can create an emulsion, or colloidal suspension of two immiscible liquids
  \item Soaps and detergents can create colloidal suspensions of oils in water (Figure 11.33)
\end{itemize}
\paragraph*{Quiz 11.4 - Molar Masses and Colloids}
\paragraph*{Homework 11.4}
\begin{itemize}
  \item 11.49: Molar mass from boiling point
  \item 11.59: Molar mass from osmotic pressure
  \item 11.73: Colloid particle size
\end{itemize}


\chapter{Kinetics}

\section{Chemical Reaction Rates}
\begin{itemize}
  \item Two reactions which we can write, but do not observe:

    \ch{2 Au(s) + 3 H2O(l) -> Au2O3(s) + 3 H2(g)} \hspace{2em} Thermodynamically non-spontaneous

    \ch{C_{diamond} -> C_{graphite}} \hspace{2em} Kinetically hindered
  \item Kinetics is the study of reaction rates (how quickly the reaction proceeds)
  \item The reaction rate is the rate of dissappearance of reactant or production of product, normalized by the stoichiometric coefficients

    $rate=\dfrac{\mathrm{d}\left[A\right]}{\nu_A\mathrm{d}t}$
  \item This is the \emph{instantaneous} rate, and in practice can only be approximated
  \item We can monitor the concentration of reactant or product over time, and calculate the average rate at different intervals
  \item Consider the reaction \ch{2 H2O2(aq) -> 2 H2O(l) + O2(g)} (Figures 12.2 and 12.3)

    $rate = -\dfrac{\mathrm{d}\left[\ch{H2O2}\right]}{2\mathrm{d}t} = \dfrac{\mathrm{d}\left[\ch{H2O}\right]}{2\mathrm{d}t} = \dfrac{\mathrm{d}\left[\ch{H2}\right]}{\mathrm{d}t}$

    $rate \approx -\dfrac{\Delta\left[\ch{H2O2}\right]}{2\Delta t} = \dfrac{\Delta\left[\ch{H2O}\right]}{2\Delta t} = \dfrac{\Delta\left[\ch{H2}\right]}{\Delta t}$
  \item Practice: Consider \ch{2 NH3(g) <=> N2(g) + 3 H2(g)} (Figure 12.5). Calculate the rate using each curve
\end{itemize}


\section{Factors Affecting Reaction Rates}
\begin{itemize}
  \item Reaction rates can vary widely from virtually instantaneous to so slow the reactiond doesn't practically happen at all
  \item Many factors affecte rates, including some that can be controlled and some that cannot
  \item The pysical state of the reactants 
  \begin{itemize}
    \item For solids, reactions occur at the surface so fine powders react more quickly than coarse ones (Figure 12.6)
    \item For heterogeneous reaction, the reaction occurs at the interface
  \end{itemize}
  \item Temperature: All reactions increase their rate as temperature increases
  \item Concentration of reactants
  \begin{itemize}
    \item Increasing reactants generally increases the rate of reaction (We won't see any exceptions in this class)
    \item Product concentration generally has no effect on reaction rates (again, no exceptions in this class)
    \item Figure 12.7 shows how degradation of statues is accelerated in areas with high \ch{H2SO4} concentration
  \end{itemize}
\item The presence of a \emph{catalyst} (more on this in section 12.7)
\end{itemize}

\section{Rate Laws}
\begin{itemize}
  \item The reaction rate can be related to reactant concentration through a \emph{rate law}
    \begin{itemize}
      \item For a generic reaction \ch{aA+bB->cC+dD}, $rate=k\left[A\right]^m\left[B\right]^n$
      \item $m$ and $n$ are called the reaction orders, and are unrelated to the stoichiometric coefficients (equations at the end of the section)
      \item $m+n$ gives the \emph{overall} reaction order
      \item $k$ is called the \emph{rate constant}, and will take different units depending on the overall reaction order (Table 12.1)
    \end{itemize}
  \item Rate laws can be determined through the \emph{Initial Rate Method}
    \begin{itemize}
      \item Do several runs of the reaction with different concentrations of reactants
      \item Measure the initial rate of reaction for each run
      \item Compare runs pairwise, choosing pairs which keep one reactant concentration constant and change the other
      \item Take the ratios of the rates, equal to the ratios of the rate laws for each condition
      \item Simplify the ratio of rate laws mathematically (just show this on the whiteboard)
      \item Calculate the value of $k$ using data from one trial (or all of them, and average the results)
      \item Practice: Work example 12.4 from the text
    \end{itemize}
\end{itemize}

\paragraph*{Quiz 12.1 - Reaction Rates}
\paragraph*{Homework 12.1}
\begin{itemize}
  \item 5: Find rates from concentration data 
  \item 7: Factors affecting rate laws
  \item 25: Initial rate method
\end{itemize}

\section{Integrated Rate Laws}
\begin{itemize}
  \item We can set the definition of reaction rate equal to the rate law: $-\dfrac{\mathrm{d}\left[A\right]}{\mathrm{d}t}=k\left[A\right]^m$
  \item Rearrange this to separate the infinitesimal terms and integrate: $\int\dfrac{\mathrm{d}\left[A\right]}{\left[A\right]^m}=\int-k\mathrm{d}t$
  \item This will integrate to give different integrated rate laws depending on the reaction order
  \item First-order
    \begin{itemize}
      \item Linear form: $\ln\left[A\right]_t=\ln\left[A\right]_0-kt$
      \item Two-point form: $\ln\left(\dfrac{\left[A\right]_t}{\left[A\right]_0}\right)=-kt$
      \item Special Half-life form: $\dfrac{A_t}{A_0}=\left(\dfrac{1}{2}\right)^{\dfrac{t}{t_{\nicefrac{1}{2}}}}$
      \item Half-life: $t_{\nicefrac{1}{2}}=\dfrac{\ln2}{k}$
      \item Linear when plotting $\ln[A]$ vs $t$, with $slope=-k$
    \end{itemize}
  \item Second-order
    \begin{itemize}
      \item Linear form: $\dfrac{1}{\left[A\right]_t}=kt+\dfrac{1}{\left[A\right]_0}$
      \item Half-life: $t_{\nicefrac{1}{2}}=\dfrac{1}{k\left[A\right]_0}$
      \item Linear when plotting $\dfrac{1}{[A]}$ vs $t$, with $slope=+k$
    \end{itemize}
  \item Zeroth-order
    \begin{itemize}
      \item Linear form: $\left[A\right]_t=-kt+\left[A\right]_0$
      \item Half-life: $t_{\nicefrac{1}{2}}=\dfrac{\left[A\right]_0}{2k}$
      \item Linear when plotting $[A]$ vs $t$, with $slope=-k$
    \end{itemize}
  \item All the above is summarized in Table 12.2 in the text
  \item Determining reaction order graphically
    \begin{itemize}
      \item Graph $[A]$, $\ln[A]$, and $\dfrac{1}{[A]}$ vs t
      \item Two will be curved, while one is straight and indicates the overall reaction order
      \item Making one reactant in excess will prove the reaction order of only the other reactant
      \item Use my prepared spreadsheet to practice determining the rate law
    \end{itemize}
\end{itemize}

\paragraph*{Quiz 12.2 - Integrated Rate Laws}
\paragraph*{Homework 12.2}
\begin{itemize}
  \item 33: Graphically determine rate law
  \item 36: Half-life from rate constant
  \item 40: Second-order half-life
  \item 46: First-order decay 
\end{itemize}

\section{Collision Theory}
% \begin{itemize}
%   \item 
% \end{itemize}

\section{Reaction Mechanisms}
% \begin{itemize}
%   \item 
% \end{itemize}

\section{Catalysis}
% \begin{itemize}
%   \item 
% \end{itemize}

\chapter{Fundamental Equilibrium Concepts}

\section{Chemical Equilibria}
% \begin{itemize}
%   \item 
% \end{itemize}

\section{Equilibrium Constants}
% \begin{itemize}
%   \item 
% \end{itemize}

\section{Shifting Equilibria: Le Ch\^atelier's Principle}
% \begin{itemize}
%   \item 
% \end{itemize}

\section{Equilibrium Calculations}
% \begin{itemize}
%   \item 
% \end{itemize}

\chapter{Acid-Base Equilibria}

\section{Brønsted-Lowry Acids and Bases}
% \begin{itemize}
%   \item 
% \end{itemize}

\section{pH and pOH}
% \begin{itemize}
%   \item 
% \end{itemize}

\section{Relative Strengths of Acids and Bases}
% \begin{itemize}
%   \item 
% \end{itemize}

\section{Hydrolysis of Salts}
% \begin{itemize}
%   \item 
% \end{itemize}

\section{Polyprotic Acids}
% \begin{itemize}
%   \item 
% \end{itemize}

\section{Buffers}
% \begin{itemize}
%   \item 
% \end{itemize}

\section{Acid-Base Titrations}
% \begin{itemize}
%   \item 
% \end{itemize}

\chapter{Equilibria of Other Reaction Classes}

\section{Precipitation and Dissolution}

\section{Lewis Acids and Bases}

\section{Coupled Equilibria}

\chapter{Thermodynamics}

\section{Spontaneity}

\section{Entropy}

\section{The Second and Third Laws of Thermodynamics}

\section{Free Energy}

\chapter{Electrochemistry}

\section{Review of Redox Chemistry}

\section{Galvanic Cells}

\section{Electrode and Cell Potentials}

\section{Potential, Fee Energy, and Equilibrium}

\section{Batteries and Fuel Cells}

\section{Corrosion}

\section{Electrolysis}

\setcounter{chapter}{20}
\chapter{Nuclear Chemistry}

\section{Nuclear Structure and Stability}

\section{Nuclear Equations}

\section{Radioactive Decay}

\section{Transmutation and Nuclear Energy}

\section{Uses of Radioisotopes}

\section{Biological Effects of Radiation}

\backmatter
\chapter{Errata}

\end{document}
