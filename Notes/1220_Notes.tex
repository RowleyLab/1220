\documentclass[12pt, openany, letterpaper]{memoir}
\usepackage{NotesStyle}
%\renewcommand\thesection{\thechapter\Alph{section}}
%\renewcommand\thesubsection{\thesection.\Numeral{subsection}}

\begin{document}
\title{CHEM 1220 Lecture Notes\\ OpenStax Chemistry 2e}
\author{Matthew Rowley}
\date{\today}
\mainmatter
\maketitle
\chapter*{Course Administrative Details}
\begin{itemize}
	\item My office hours
	\item Intro to my research
	\item Introductory Quiz
	\item Grading details
	      \begin{itemize}
		      \item Exams - 40, Final - 20, Online Homework - 15, Book Homework - 15, Quizzes - 10
		      \item Online homework
		      \item Frequent quizzes
	      \end{itemize}
	\item Importance of reading and learning on your own
	\item Learning resources
	      \begin{itemize}
		      \item My Office Hours
		      \item Tutoring services - \href{https://www.suu.edu/academicsuccess/tutoring/}{https://www.suu.edu/academicsuccess/tutoring/}
	      \end{itemize}
	\item Show how to access Canvas
	      \begin{itemize}
		      \item Calendar, Grades, Modules, etc.
		      \item Quizzes
		      \item Textbook
	      \end{itemize}
	\item Introduction to chemistry
	      \begin{itemize}
		      \item Ruby fluorescence
		      \item Levomethamphetamine
		      \item Rubber band elasticity
		      \item Structure of the periodic table
		      \item Salt on ice and purifying hydrogen peroxide
	      \end{itemize}
\end{itemize}

\setcounter{chapter}{-1}
\chapter{1210 Review}

There is a whole semester of material from 1210, and these are only the topics which are \emph{most} important for success in 1220

\begin{itemize}
	\item Composition of atoms and ions (protons, neutrons and electrons)
	\item Chemical formulas and names
	\begin{itemize}
		\item Formulas and molar masses
		\item Polyatomic ion names
		\item Naming ionic compounds
		\item Naming binary molecular compounds
		\item Naming acids
	\end{itemize}
	\item Balancing molecular equations
	\item Solubility rules
	\item Fundamentals of acid/base chemistry
	\item Measurements vs. chemistry
	\begin{itemize}
		\item Converting from measurements to moles and back
		\item Stoichiometry and predicting amounts
		\item Limiting reactants
	\end{itemize}
	\item Enthalpy of reaction and heat equations
	\item Lewis structures
\end{itemize}

\setcounter{chapter}{9}
\chapter{Liquids and Solids}

\section{Intermolecular Forces}

\section{Properties of Liquids}

\section{Phase Transitions}

\section{Phase Diagrams}

\section{The Solid State of Matter}

\section{Lattice Structures in Crystalline Solids}

\chapter{Solutions and Colloids}

\section{The Dissolution Process}

\section{Electrolytes}

\section{Solubilty}

\section{Colligative Properties}

\section{Colloids}

\chapter{Kinetics}

\section{Chemical Reaction Rates}

\section{Factors Affecting Reaction Rates}

\section{Rate Laws}

\section{Integrated Rate Laws}

\section{Collision Theory}

\section{Reaction Mechanisms}

\section{Catalysis}

\chapter{Fundamental Equilibrium Concepts}

\section{Chemical Equilibria}

\section{Equilibrium Constants}

\section{Shifting Equilibria: Le Ch\^atelier's Principle}

\section{Equilibrium Calculations}

\chapter{Acid-Base Equilibria}

\section{Brønsted-Lowry Acids and Bases}

\section{pH and pOH}

\section{Relative Strengths of Acids and Bases}

\section{Hydrolysis of Salts}

\section{Polyprotic Acids}

\section{Buffers}

\section{Acid-Base Titrations}

\chapter{Equilibria of Other Reaction Classes}

\section{Precipitation and Dissolution}

\section{Lewis Acids and Bases}

\section{Coupled Equilibria}

\chapter{Thermodynamics}

\section{Spontaneity}

\section{Entropy}

\section{The Second and Third Laws of Thermodynamics}

\section{Free Energy}

\chapter{Electrochemistry}

\section{Review of Redox Chemistry}

\section{Galvanic Cells}

\section{Electrode and Cell Potentials}

\section{Potential, Fee Energy, and Equilibrium}

\section{Batteries and Fuel Cells}

\section{Corrosion}

\section{Electrolysis}

\setcounter{chapter}{20}
\chapter{Nuclear Chemistry}

\section{Nuclear Structure and Stability}

\section{Nuclear Equations}

\section{Radioactive Decay}

\section{Transmutation and Nuclear Energy}

\section{Uses of Radioisotopes}

\section{Biological Effects of Radiation}

\backmatter
\chapter{Errata}

\end{document}
