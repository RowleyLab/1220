\documentclass[12pt, openany, letterpaper]{memoir}
\usepackage{NotesStyle}
%\renewcommand\thesection{\thechapter\Alph{section}}
%\renewcommand\thesubsection{\thesection.\Numeral{subsection}}

\begin{document}
\title{CHEM 1220 Lecture Notes\\ OpenStax Chemistry 2e}
\author{Matthew Rowley}
\date{\today}
\mainmatter
\maketitle
\chapter*{Course Administrative Details}
\begin{itemize}
	\item My office hours
	\item Intro to my research
	\item Introductory Quiz
	\item Grading details
	      \begin{itemize}
		      \item Exams - 40, Final - 15, Online Homework - 15, Book Homework - 15, Quizzes - 15
		      \item Online homework
		      \item Frequent quizzes
	      \end{itemize}
	\item Importance of reading and learning on your own
	\item Learning resources
	      \begin{itemize}
		      \item My Office Hours
		      \item Tutoring services - \href{https://www.suu.edu/academicsuccess/tutoring/}{https://www.suu.edu/academicsuccess/tutoring/}
	      \end{itemize}
	\item Show how to access Canvas
	      \begin{itemize}
		      \item Calendar, Grades, Modules, etc.
		      \item Quizzes
		      \item Textbook
	      \end{itemize}
	\item Introduction to chemistry
	      \begin{itemize}
		      \item Ruby fluorescence
		      \item Levomethamphetamine
		      \item Rubber band elasticity
		      \item Structure of the periodic table
		      \item Salt on ice and purifying hydrogen peroxide
	      \end{itemize}
\end{itemize}

\setcounter{chapter}{-1}
\chapter{1210 Review}

There is a whole semester of material from 1210, and these are only the topics which are \emph{most} important for success in 1220

\begin{itemize}
	\item Composition of atoms and ions (protons, neutrons and electrons)
	\item Chemical formulas and names
	\begin{itemize}
		\item Formulas and molar masses
		\item Polyatomic ion names
		\item Naming ionic compounds
		\item Naming binary molecular compounds
		\item Naming acids
	\end{itemize}
	\item Balancing molecular equations
	\item Solubility rules
	\item Fundamentals of acid/base chemistry
	\item Measurements in chemistry
	\begin{itemize}
		\item Converting from measurements to moles and back
		\item Stoichiometry and predicting amounts
		\item Limiting reactants
	\end{itemize}
	\item Enthalpy of reaction and heat equations
	\item Lewis structures
\end{itemize}
\paragraph*{CHEM 1210 Review Quiz}

\setcounter{chapter}{9}
\chapter{Liquids and Solids}

\section{Intermolecular Forces}
\begin{itemize}
  \item Many physical properties of solids, liquids, and gases can be explained by the strength of attractive forces between particles (Figure 10.5)
  \item Phase changes happen due to the interplay between kinetic energy and intermolecular forces (Figure 10.2)
  \item Pressure can also play a role in phase changes, as discussed later
  \item These \emph{intermolecular forces} come in different varieties
  \begin{itemize}
    \item Dispersion Forces Non-polar molecules, impacted by polarizability, molecular weight, and surface area
    \begin{itemize}
      \item Dominant in non-polar molecules
      \item Created by induced dipoles (Figure 10.6)
      \item Impacted by polarizability (Table 10.1)
      \item Impacted by molecular weight (hydrocarbons from methane to wax)
      \item Impacted by molecule shape (Figure 10.7 compares the boiling points of pentane isomers)
    \end{itemize}
    \item Dipole-Dipole Forces
    \begin{itemize}
      \item Dominant in polar molecules
      \item Results from attraction between permanent dipoles (Figure 10.9)
    \end{itemize}
  \item Hydrogen Bonding
    \begin{itemize}
      \item Dominant only in molecules capable of hydrogen bonding
      \item Must contain a hydrogen-donor atom (H attached to N, O, or F)
      \item Must contain a hydrogen-acceptor atom (lone pair of electrons attached to N, O, or F)
      \item Hydrogen bonds are more than just particularly strong dipole-dipole forces. They have strong directionality according to VSEPR
      \item Figures 10.10, 10.14, and other figures on the Internet show water, DNA, and proteins all organized by hydrogen bonds
      \item Figures 10.11 and 10.12 illustrate how much hydrogen bonds exceed dipole-dipole forces in strength
    \end{itemize}
  \end{itemize}
\end{itemize}

\section{Properties of Liquids}
\begin{itemize}
  \item Viscosity is a fluid's resistance to flow
  \begin{itemize}
    \item We intuitively know that both water and honey flow\ldots but at very different rates
    \item Viscosity is proportional to the strength of intermolecular forces (high IF = high viscosity)
    \item As temperature increases, kinetic energy is able to overcome intermolecular forces and viscosity decreases
    \item Table 10.2 gives the viscosities of some common substances (note the unusual units!)
  \end{itemize}
  \item Surface tension is a force which minimizes a fluid's surface area
  \begin{itemize}
    \item Cohesive vs. adhesive forces
    \item Bulk molecules have lower energy than surface molecules due to being \emph{surrounded} by cohesive forces (Figure 10.16)
    \item Figure 10.17 illustrates a waterbug supporting itself on water surface tension
    \item Surface tension is often in conflict with gravity and other forces, making most liquids rounded but not perfect spheres
    \item Surface tension is proportional to intermolecular forces (Table 10.3)
    \item Surface tension can be strongly affected by addition of certain solutes, called surfactants
  \end{itemize}
  \item Capillary action is a force between a fluid and narrow channels or capillaries of solid materials
  \begin{itemize}
    \item Due to adhesive forces with the solid, liquids will be drawn up (or, less often, pushed down) a capillary
    \item Figure 10.19 shows how paper towels are made to maximize capillary action, so they soak up water-based spills
    \item The top of the liquid (called the meniscus) will curve differently depending on the reletive strength of cohesive and adhesive forces (Figure 10.18)
    \item Figure 10.20 shows capillary action in a variety of situations, including capillary repulsion
    \item Remember that when measuring volumes, convention is to read the \emph{bottom} of the meniscus regardless of how it curves
    \item Don't worry about the formula given here
  \end{itemize}
\end{itemize}
\paragraph*{Quiz 10.1 - Intermolecular Forces and Liquid Properties}
\paragraph*{Homework 10.1}
\begin{itemize}
  \item 10.11: Predicting trends in boiling points
  \item 10.21: Identifying intermolecular forces
  \item 10.25: Affect of temperature on viscosity
\end{itemize}

\section{Phase Transitions}
\begin{itemize}
  \item Vaporization and condensation are the transitions between liquid and gas phases
  \begin{itemize}
    \item The enthalpy of vaporization $\left(\Delta H_{vap}\right)$ is the energy required to transition from liquid to gas phase
    \item Enthalpy of condensation is the opposite $\Delta H_{con} = - \Delta H_{vap}$
    \item In a closed volume, these processes will reach a \emph{dynamic equilibrium}
    \item The partial pressure of the liquid at this equilibrium state is called its \emph{vapor pressure} (Figure 10.22)
    \item Higher intermolecular forces lead to lower vapor pressures
    \item Higher temperatures increase the vapor pressure due to increased kinetic energy (Figure 10.23)
  \end{itemize}
  \item Boiling points
  \begin{itemize}
    \item Figure 10.24 shows vapor pressure curves and the normal boiling points of several liquids
    \item Boiling points generally depend on the pressure (pressure cookers, boiling water to freezing, etc.)
    \item The Clausius-Clapeyron equation defines these curves (Note the rearrangments I've made)
      \\ $P=Ae^{\nicefrac{-\Delta H_{vap}}{RT}}$ \hspace{2em} 
      $\ln P = -\dfrac{\Delta H_{vap}}{RT}+\ln A$ \hspace{2em}
      $\ln\left(\dfrac{P_2}{P_1}\right)=-\dfrac{\Delta H_{vap}}{R}\left(\dfrac{1}{T_2}-\dfrac{1}{T_1}\right)$ 
  \end{itemize}
  \item Fusion (melting), freezing, sublimation, and deposition all have their enthalpies and transition temperatures
  \item These enthalpies are state functions, such that $\Delta H_{sub} = \Delta H_{fus}+\Delta H_{vap}$ (Figure 10.28)
  \item Heating and Cooling curves
  \begin{itemize}
    \item When heat is added to a system, it will either cause a phase change, or a change in temperature
    \item For phase changes, $q=n\Delta H_{change}$
    \item For temperature changes, $q=mc\Delta T$, where $c$ is the specific heat for that substance and phase
    \item Sometimes $\Delta H_{change}$ is given as a -per gram value, and sometimes $c$ is given as a -per mole value, but usually not :(
    \item Figure 10.29 shows a typical heating curve (Work example 10.10 in the text)
  \end{itemize}
\end{itemize}
\paragraph*{Quiz 10.2 - Heating Curves}
\paragraph*{Homework 10.2}
\begin{itemize}
  \item 10.31: Temperature during a phase transition
  \item 10.39: Definition of normal boiling point
  \item 10.51: Heating curve problem
\end{itemize}

\section{Phase Diagrams}
\begin{itemize}
  \item The stable phase at different temperatures and pressures is best illustrated with a phase diagram (Figures 10.30, 10.31)
  \item We can tell at a glance what transitions might occur as we increase or decrease either the temperature or pressure
  \item Note that at some pressures, sublimation may occur instead of fusion
  \item The triple point is a unique point where liquid, solid, and gas can all exist at equilibrium (contrast with a glass of icy water on a humid day)
  \item The critical point is where the distinction between liquid and solid phases disappears
  \item Figure 10.34 shows the phase diagram of \ch{CO2}
  \item Supercritical fluids exhibit some interesting properties, and are often great solvents (Nile Blue Youtube video)
  \item Critical points vary widely depending on the intermolecular forces, and other factors (Table in text)
\end{itemize}
\paragraph*{Quiz 10.3 - Phase Diagrams}
\paragraph*{Homework 10.3}
\begin{itemize}
  \item 10.55: Trajectories on a phase diagram
  \item 10.57: Determining state on a phase diagram
  \item 10.63: Identifying phases on a blank phase diagram
\end{itemize}

\section{The Solid State of Matter}
\begin{itemize}
  \item Solids can be divided into \emph{crystalline} and \emph{amorphous} based on their structure at atomic scales
  \item Figure 10.37 shows the difference generally, Figure 10.38 shows crystalline and amorphous \ch{SiO2}
  \item Amorphous solids will not exhibit a sharp fusion transition temperature, but will instead grow soft and maleable over a temperature range
  \item Crystalline solids are diverse but always show long-range repeating order in their structure
  \begin{itemize}
    \item Ionic solids (Figure 10.39) have high melting points, cleave along planes, and conduct electricity only in the liquid phase
    \item Metallic solids (Figure 10.40) have mostly high melting points, are maleable and ductile, and conduct electricty and heat well
    \item Covalent network solids (Figure 10.41) have very high melting points and are electrical insulators
    \item Molecular solids (Figure 10.42) Have low to very low melting points and are electrical insulators
    \item Crystalline solid properties are summarized in Table 10.4
  \end{itemize}
  \item Even crystalline solids do not have perfect structure. Various types of defects are illustrated in Figure 10.45
\end{itemize}
\paragraph*{Quiz 10.4 - Classifying Solids}
\paragraph*{Homework 10.4}
\begin{itemize}
  \item 10.69: Classify solids by formulas
  \item 10.71: Classify solids by properties
\end{itemize}

\section{Lattice Structures in Crystalline Solids}
\begin{itemize}
  \item The structure of a crystalline solid is represented by a \emph{unit cell}, the smallest repeatable unit of the structure
  \item Sometimes this microscopic structure is evidently manifested on macroscopic scales, but sometimes it isn't
  \item Unit cells are defined by lattice points that often lie at the center of certain atoms, and the cell edges often cut atoms in half, quarter, etc.
  \item Unit cells of metals
  \begin{itemize}
    \item For metals, we should keep track of the quantity of atoms in a unit cell, the coordination number, and the relationship between the atomic radius and unit cell edge length
    \item Simple cubic (Figure 10.49) 1 atom, Coordination=6, $l=2r$
    \item Body-centered cubic (Figure 10.51) 2 atoms, Coordination=8, $l=\frac{4}{\sqrt{3}}r$
    \item Face-centered cubic (Figure 10.52) 4 atoms, Coordination=12, $l=\sqrt{8}r$
    \item Figure 10.54 shows hexagonal closest packed and cubic closest packed structures
    \item Find the radius of a gold atom, which has fcc structure and a density of $19.283\nicefrac{g}{cm^3}$ ($136pm$)
    \item Find the density of polonium, which has sc structure and an atomic radius of $140pm$ ($9.20\nicefrac{g}{cm^3}$)
    \item Figure 10.56 shows many non-cubic structures which are common as well
  \end{itemize}
  \item Unit cells of ionic compounds
  \begin{itemize}
    \item Anions are generally larger than cations, so ionic lattice points are generally the centers of anions
    \item Cations occupy holes in the anionic lattice (Figures 10.57 and 10.58)
    \item Unit cells of ionic structures share names with the metallic cells but look different because of the cations
    \item Simple cubic (Figure 10.59)
    \item Face-centered cubic (rock salt structure) (Figure 10.60)
    \item Zinc blende (Figure 10.61)
    \item Find the ionic bond length for \ch{NaCl} which has rock salt structure and denisty of $2.17\nicefrac{g}{cm^3}$ ($l=564pm$)
  \end{itemize}
  \item Crystal structure is determined through X-ray crystallography
  \begin{itemize}
    \item X-rays reflected off a crystal surface can combine destructively or constructively to produce an interference pattern (Figure 10.63)
    \item The X-rays will take different pathlengths depending on the angle of the X-ray beam and the crystal lattice constant (Figure 1.64)
    \item An experimental setup and actual diffractogram are shown in Figures 10.65 and 10.66
    \item We have a powerful X-ray instrument here at SUU
  \end{itemize}
\end{itemize}
\paragraph*{Quiz 10.5 - Unit Cells}
\paragraph*{Homework 10.5}
\begin{itemize}
  \item 10.77: Coordination number
  \item 10.81: Density from lattice constant
  \item 10.85: Packing efficiency and density
\end{itemize}

\chapter{Solutions and Colloids}

\section{The Dissolution Process}
\begin{itemize}
  \item Some vocabulary: \emph{Solution}, \emph{Solvent}, \emph{Solute}, and \emph{solvation}
  \item Table 11.1 shows many different types of solutions, with different phases of solvent and solute
  \item Molecular compounds dissolve to form one solute:

    \ch{C6H12O6(s) -> C6H12O6(aq)}
  \item Ionic compounds will dissolve into individual ions:

    \ch{Na2SO3(s) -> Na2SO3(aq) -> 2 Na^+(aq) + SO3^{2-}(aq)}
  \item Dissolving soluble compounds is a thermodynamically \emph{spontaneous} process
  \begin{itemize}
    \item Spontaneity is covered in more detail in chapter 16
    \item Solvation mixes solvent and solute, increasing the system \emph{entropy} (Figure 11.3)
    \item Solvation can be either \emph{exothermic} (favors spontaneity) or \emph{endothermic} (hampers spontaneity) depending on the strength of solvent-solvent, solute-solute, and solvent-solute intermolecular forces (Figure 11.4)
    \item Demonstration, dissolving \ch{NaOH(s)} and \ch{NH4NO3(s)} in water (Don't overdo the \ch{NaOH}!)
    \item When solvation has $\Delta H \approx 0$, the result is an \emph{ideal solution}, whose properties best match simple laws
  \end{itemize}
\end{itemize}

\section{Electrolytes}
\begin{itemize}
  \item Electrolytes will yield ions when dissolved in water, yielding a solution which conducts electricity (Figure 11.6)
  \begin{description}
    \item[Non-electrolytes:] Do not yield ions at all when dissolved (Most molecular compounds)
    \item[Strong electrolytes:] Produce a large (stoichiometric) amount of ions when dissolved (Soluble ionic compounds and \emph{strong} acids/bases)
    \item[Weak electrolytes:] Produce a smaller amount of ions when dissolved (\emph{weak} acids/bases)
  \end{description}
  \item Ionic electrolytes produce ions by directly \emph{dissociating} into their cations and anions (Figure 11.7)
  \item Molecular electrolytes produce ions by reacting with the solvent or other molecules

    \ch{NH3(aq) + H2O(l) <=> NH4^+(aq) + OH^-(aq)}
\end{itemize}

\section{Solubilty}
\begin{itemize}
  \item Table 4.1 gave rules to predict if an ionic compound is soluble or insoluble, but in reality solubilities lie on a spectrum
  \item In Chapter 15, we will explore solubility with mathematic rigor. For now, we will focus on trends and factors affecting solubility
  \item Solubility is a type of reaction governed by \emph{equilibrium}
  \begin{description}
    \item[Unsaturated] solutions have not yet reached their limit of how much solute they can dissolve
    \item[Saturated] solutions have met their solubility limit and are in equilibrium. You can recognize a saturated solution by the presence of undissolved solute in contact with the solution
    \item[Supersaturated] solutions have exceeded their solubility limit. This situation is only \emph{metastable} and usually contrived by quick changes in temperature or volume (``Jeremy Krug'' Youtube video of supersaturated \ch{NaCH3CO2})
  \end{description}
  \item Solutions of gases in liquids
  \begin{itemize}
    \item Gas-in-liquid solvation is always exothermic and solubility depends primarily on solvent-solute interactions
    \item Solubility is decreases as temperature rises (Figures 11.8 and 11.9)
    \item Solubility also depends on the gas partial pressure, according to Henry's law. Figure 11.8 gives $k_H$, and Figure 11.10 illustrates how to use Henry's law to supersaturate a solution (carbonation!)

      $C_{gas}=k_HP_{gas}$
  \end{itemize}
  \item Solutions of liquids in liquids
  \begin{itemize}
    \item Miscible liquids are infinitely soluble in each other (mix in any ratio)
    \item Immiscible liquids have very low solubility in each other, and separate to form layers. Oil and water (Figure 11.14) are a classic example of immiscible liquids and illustrate the axiom that ``like dissolves like'' because their intermolecular forces are so different
    \item Partially miscible liquids will form two layers when mixed, but each layer contains significant amounts of the other solute liquid
  \end{itemize}
  \item Solutions of solids in liquids
  \begin{itemize}
    \item Figure 11.6 shows the temperature dependance of solubility for several solids
    \item \emph{Exothermic} $\Delta H_{solv}$ leads to lower solubility at higher temperatures
    \item \emph{Endothermic} $\Delta H_{solv}$ leads to higher solubility at higher temperatures
  \end{itemize}
\end{itemize}
\paragraph*{Quiz 11.1 - The Solvation Process}
\paragraph*{Homework 11.1}
\begin{itemize}
  \item 11.3: Energetics of solvation
  \item 11.9: Rule of like dissolves like
  \item 11.13: Classifying electrolytes
  \item 11.23: Henry's law
\end{itemize}

\section{Colligative Properties}
\begin{itemize}
  \item Colligative properties of solutions depend on the \emph{amount} of solute present, regardless of the chemical identity of the solutes
  \item Some colligative properties depend on less common units of concentration
  \begin{itemize}
    \item Recall molarity from chapter 3
    \item \emph{Mass \%} is $\dfrac{m_{solute}}{m_{total}}\times 100\%$
    \item \emph{Mole fraction} is $\chi_A=\dfrac{n_A}{n_{total}}$
    \item \emph{Molality} is $m=\dfrac{moles_{solute}}{kg_{solvent}}$
    \item Practice interconverting between these units: $\chi_{\ch{C6H12O6}}=0.25$ in aqueous solution
    \item For electrolytes, we will also need the van't Hoff factor, $i = $ \# of particles produce on solvation
  \end{itemize}
\end{itemize}

\paragraph*{Quiz 11.2 - Concentrations}
\paragraph*{Homework 11.2}
\begin{itemize}
  \item 11.19: \% by mass and solubility
  \item 11.31: Mole fraction
  \item 11.39: Molality
\end{itemize}

\paragraph*{Back to Section 11.4  Colligative Properties}
\begin{itemize}
  \item Vapor pressure lowering
  \begin{itemize}
    \item Figure 11.8 illustrates why solutes lower the vapor pressure of the solvent
    \item Rault's law: $P_A=\chi_AP^\star_A$
    \item If the solute is a liquid, we can apply Rault's law to the solute as well $P_{total}=\chi_AP^\star_A+\chi_BP^\star_B$
    \item This gives a different composition of the gas phase from the liquid phase, allowing for purification through distillation (Figures 11.19 and 11.20)
  \end{itemize}
  \item Changes in phase tranistion temperatures
    \begin{itemize}
      \item Boiling point elevation is a consequence of vapor pressure lowering
      \item Freezing point depression follows a similar forumula 
      \item $\Delta T_{f/b}=iK_{f/b}m$
      \item Table 11.2 gives $T_b$, $K_b$, $T_f$, and $K_f$ for several substances
      \item These effects manifest on a phase diagram like in Figure 11.23
    \end{itemize}
  \item Osmotic pressure
  \begin{itemize}
    \item Figure 11.24 shows how water will flow across a selectively permeable membrane via osmosis
    \item Figure 11.25 shows how and applied pressure can reverse this process and purify water
    \item $\pi=\dfrac{inRT}{V} = iMRT$
    \item Figure 11.27 shows how blood salinity can impact the health of red blood cells (isotonic, hypertonic, hypotonic, hemolysis, crenation)
  \end{itemize}
  \item Measuring colligative properties can give the molar mass of an unknown, as long as we know the van't Hoff factor
\end{itemize}
\paragraph*{Quiz 11.3 - Colligative Properties}
\paragraph*{Homework 11.3}
\begin{itemize}
  \item 11.45: Freezing point depression
  \item 11.61: Osmotic pressure
  \item 11.65: Vapor pressures of mixtures
\end{itemize}

\section{Colloids}
\begin{itemize}
  \item Colloids occupy the blurry boundry region between homogeneous and heterogeneous mixtures (Figure 11.29)
  \item Colloids can be identified by several properties:
  \begin{itemize}
    \item Particle size is on a range of tens- to hundreds- of nanometers
    \item Particles will not settle out on their own under the influence of gravity
    \item Particles will scatter light, called the Tyndall Effect (Figure 11.30)
  \end{itemize}
  \item Table 11.4 gives some examples of colloids in various phases
  \item Emulsifying agents can create an emulsion, or colloidal suspension of two immiscible liquids
  \item Soaps and detergents can create colloidal suspensions of oils in water (Figure 11.33)
\end{itemize}
\paragraph*{Quiz 11.4 - Molar Masses and Colloids}
\paragraph*{Homework 11.4}
\begin{itemize}
  \item 11.49: Molar mass from boiling point
  \item 11.59: Molar mass from osmotic pressure
  \item 11.73: Colloid particle size
\end{itemize}


\chapter{Kinetics}

\section{Chemical Reaction Rates}
\begin{itemize}
  \item Two reactions which we can write, but do not observe:

    \ch{2 Au(s) + 3 H2O(l) -> Au2O3(s) + 3 H2(g)} \hspace{2em} Thermodynamically non-spontaneous

    \ch{C_{diamond} -> C_{graphite}} \hspace{2em} Kinetically hindered
  \item Kinetics is the study of reaction rates (how quickly the reaction proceeds)
  \item The reaction rate is the rate of dissappearance of reactant or production of product, normalized by the stoichiometric coefficients

    $rate=\dfrac{\mathrm{d}\left[A\right]}{\nu_A\mathrm{d}t}$
  \item This is the \emph{instantaneous} rate, and in practice can only be approximated
  \item We can monitor the concentration of reactant or product over time, and calculate the average rate at different intervals
  \item Consider the reaction \ch{2 H2O2(aq) -> 2 H2O(l) + O2(g)} (Figures 12.2 and 12.3)

    $rate = -\dfrac{\mathrm{d}\left[\ch{H2O2}\right]}{2\mathrm{d}t} = \dfrac{\mathrm{d}\left[\ch{H2O}\right]}{2\mathrm{d}t} = \dfrac{\mathrm{d}\left[\ch{H2}\right]}{\mathrm{d}t}$

    $rate \approx -\dfrac{\Delta\left[\ch{H2O2}\right]}{2\Delta t} = \dfrac{\Delta\left[\ch{H2O}\right]}{2\Delta t} = \dfrac{\Delta\left[\ch{H2}\right]}{\Delta t}$
  \item Practice: Consider \ch{2 NH3(g) <=> N2(g) + 3 H2(g)} (Figure 12.5). Calculate the rate using each curve
\end{itemize}


\section{Factors Affecting Reaction Rates}
\begin{itemize}
  \item Reaction rates can vary widely from virtually instantaneous to so slow the reactiond doesn't practically happen at all
  \item Many factors affecte rates, including some that can be controlled and some that cannot
  \item The pysical state of the reactants 
  \begin{itemize}
    \item For solids, reactions occur at the surface so fine powders react more quickly than coarse ones (Figure 12.6)
    \item For heterogeneous reaction, the reaction occurs at the interface
  \end{itemize}
  \item Temperature: All reactions increase their rate as temperature increases
  \item Concentration of reactants
  \begin{itemize}
    \item Increasing reactants generally increases the rate of reaction (We won't see any exceptions in this class)
    \item Product concentration generally has no effect on reaction rates (again, no exceptions in this class)
    \item Figure 12.7 shows how degradation of statues is accelerated in areas with high \ch{H2SO4} concentration
  \end{itemize}
\item The presence of a \emph{catalyst} (more on this in section 12.7)
\end{itemize}

\section{Rate Laws}
\begin{itemize}
  \item The reaction rate can be related to reactant concentration through a \emph{rate law}
    \begin{itemize}
      \item For a generic reaction \ch{aA+bB->cC+dD}, $rate=k\left[A\right]^m\left[B\right]^n$
      \item $m$ and $n$ are called the reaction orders, and are unrelated to the stoichiometric coefficients (equations at the end of the section)
      \item $m+n$ gives the \emph{overall} reaction order
      \item $k$ is called the \emph{rate constant}, and will take different units depending on the overall reaction order (Table 12.1)
    \end{itemize}
  \item Rate laws can be determined through the \emph{Initial Rate Method}
    \begin{itemize}
      \item Do several runs of the reaction with different concentrations of reactants
      \item Measure the initial rate of reaction for each run
      \item Compare runs pairwise, choosing pairs which keep one reactant concentration constant and change the other
      \item Take the ratios of the rates, equal to the ratios of the rate laws for each condition
      \item Simplify the ratio of rate laws mathematically (just show this on the whiteboard)
      \item Calculate the value of $k$ using data from one trial (or all of them, and average the results)
      \item Practice: Work example 12.4 from the text
    \end{itemize}
\end{itemize}

\paragraph*{Quiz 12.1 - Reaction Rates}
\paragraph*{Homework 12.1}
\begin{itemize}
  \item 5: Find rates from concentration data 
  \item 7: Factors affecting rate laws
  \item 25: Initial rate method
\end{itemize}

\section{Integrated Rate Laws}
\begin{itemize}
  \item We can set the definition of reaction rate equal to the rate law: $-\dfrac{\mathrm{d}\left[A\right]}{\mathrm{d}t}=k\left[A\right]^m$
  \item Rearrange this to separate the infinitesimal terms and integrate: $\int\dfrac{\mathrm{d}\left[A\right]}{\left[A\right]^m}=\int-k\mathrm{d}t$
  \item This will integrate to give different integrated rate laws depending on the reaction order
  \item First-order
    \begin{itemize}
      \item Linear form: $\ln\left[A\right]_t=\ln\left[A\right]_0-kt$
      \item Two-point form: $\ln\left(\dfrac{\left[A\right]_t}{\left[A\right]_0}\right)=-kt$
      \item Special Half-life form: $\dfrac{A_t}{A_0}=\left(\dfrac{1}{2}\right)^{\dfrac{t}{t_{\nicefrac{1}{2}}}}$
      \item Half-life: $t_{\nicefrac{1}{2}}=\dfrac{\ln2}{k}$
      \item Linear when plotting $\ln[A]$ vs $t$, with $slope=-k$
    \end{itemize}
  \item Second-order
    \begin{itemize}
      \item Linear form: $\dfrac{1}{\left[A\right]_t}=kt+\dfrac{1}{\left[A\right]_0}$
      \item Half-life: $t_{\nicefrac{1}{2}}=\dfrac{1}{k\left[A\right]_0}$
      \item Linear when plotting $\dfrac{1}{[A]}$ vs $t$, with $slope=+k$
    \end{itemize}
  \item Zeroth-order
    \begin{itemize}
      \item Linear form: $\left[A\right]_t=-kt+\left[A\right]_0$
      \item Half-life: $t_{\nicefrac{1}{2}}=\dfrac{\left[A\right]_0}{2k}$
      \item Linear when plotting $[A]$ vs $t$, with $slope=-k$
    \end{itemize}
  \item All the above is summarized in Table 12.2 in the text
  \item Determining reaction order graphically
    \begin{itemize}
      \item Graph $[A]$, $\ln[A]$, and $\dfrac{1}{[A]}$ vs t
      \item Two will be curved, while one is straight and indicates the overall reaction order
      \item Making one reactant in excess will prove the reaction order of only the other reactant
      \item Use my prepared spreadsheet to practice determining the rate law
    \end{itemize}
\end{itemize}

\paragraph*{Quiz 12.2 - Integrated Rate Laws}
\paragraph*{Homework 12.2}
\begin{itemize}
  \item 33: Graphically determine rate law
  \item 36: Half-life from rate constant
  \item 40: Second-order half-life
  \item 46: First-order decay 
\end{itemize}

\section{Collision Theory}
\begin{itemize}
	\item Collision theory explains reaction rates in terms of molecular collisions
	\begin{itemize}
		\item Reactions only occur when reactant molecules encounter each other in collisions, but not all collisions will lead to a reaction
		\item Some collisions happen in the wrong orientation to lead to reaction (Figure 12.13)
		\item Some collisions don't have enough energy to overcome the activation energy barrier
  \end{itemize}
	\item Reaction coordinate diagrams show how the energy changes over the course of a reaction
	\begin{itemize}
		\item Figure 12.14 shows a typical reaction coordinate diagram
		\item For simple reactions the x-axis can represent actual measurements, like bond lengths
		\item Generally, the x-axis just represents progress in the reaction from reactants to products
		\item The diagram shows if the reaction is exothermic or endothermic
		\item The highest energy point is called the \emph{transition state}
		\item At the transition state, reactant bonds are nearly broken but product bonds have barely started to form
		\item The activation energy is the energy required to reach the transition state
	\end{itemize}
  \item These considerations are summarized by the Arrhenius Equation: $k=Ae^{-\nicefrac{E_a}{RT}}$
  \begin{itemize}
		\item $k$ is the rate constant from the rate law
		\item $A$ is the frequency factor, and it includes both the rate of collisions, and the fraction of those collisions which lead to reaction
		\item $A$ is dependent only weakly temperature, so we'll assume that it is constant
		\item $E_a$ is the activation energy in $\dfrac{J}{mol}$, so we should use $R=8.314\dfrac{J}{mol~K}$
		\item The exponential term is called a Boltzmann factor, and gives the fraction of collisions which have enough energy
		\item Figure 12.15 shows how temperature affects the kinetic energy of collisions
	\end{itemize}
	\item We can use the Arrhenius Equation to measure the activation energy
	\begin{itemize}
		\item Take the natural log of both sides of the Arrhenius equation
		\item $\ln k=\ln\left(Ae^{-\nicefrac{E_a}{RT}}\right)$ ~~ becomes ~~ $\ln k = \ln A - \dfrac{E_a}{RT}$
		\item If we plot the $\ln k$ at different temperatures against $\dfrac{1}{T}$, we get a linear equation
		\item The slope of the line is $-\dfrac{E_a}{RT}$ and the intercept is $\ln A$
		\item We can also create the two-point form of this equation: $\ln\left(\dfrac{k_2}{k_1}\right)=-\dfrac{E_a}{R}\left(\dfrac{1}{T_2}-\dfrac{1}{T_1}\right)$ (Different from the text)
		\item Measure the reaction rate and get the rate constant at two or more temperatures
		\item Put the values into the two-point equation to get $E_a$		
	\end{itemize}
	\item The decomposition of \ch{HI} proceeds as follows: \ch{2 HI(g) -> H2(g) + I2(g)}
	
	At $655~K$ the rate constant is $8.15\times10^{-8}\dfrac{1}{M~s}$ and at $705~K$ the rate constant is $1.39\times10^{-6}\dfrac{1}{M~s}$
	
	Determine the activation energy and frequency factor for this reaction $\left(218\dfrac{kJ}{mol}~\mathrm{and}~1.91\times10^{10}\dfrac{1}{M~s}\right)$
\end{itemize}

\paragraph*{Quiz 12.3 - Arrhenius Equation}
\paragraph*{Homework 12.3}
\begin{itemize}
  \item 50: Factors in collision theory
  \item 58: Determining the frequency factor
  \item 62: Determining the activation energy
\end{itemize}

\section{Reaction Mechanisms}
\begin{itemize}
  \item My figure about how we often don't think about the \emph{mechanics} of how chemical reactions proceed
	\item Chemical reaction \emph{can} happen in just one step, but often proceed in two or more distinct steps
	\item The details of how a reaction actually proceeds is called the Reaction Mechanism
	\item Each step in the mechanism cannot be broken down or simplified further, and is called an \emph{elementary step} 
	\item Elementary steps involve either the spontaneous decomposition of a single molecule, or an encounter between two molecules
	\item Consider the following reaction: \ch{NO2(g) + CO(g) -> NO(g) + CO2(g)}
	
	Elementary Step 1: \ch{NO2(g) + NO2(g) -> NO(g) + NO3(g)} \hspace{1em} SLOW
	
	Elementary Step 2: \ch{NO3(g) + CO(g) -> NO2(g) + CO2(g)} \hspace{1em} FAST
	\item The two steps add up to the total equation
	\item \ch{NO3(g)} is produced in the first step, then consumed in the second step, so it never shows up in the overall reaction
	\item This makes \ch{NO3(g)} an \emph{intermediate}
	\item Intermediates are different from transition states because they are energetically stable (minimum in the reaction coordinate diagram)
	\item Intermediates may sometimes be observed directly in the course of the reaction, or they may be so dilute or so short-lived that they cannot be observed
	\item A reaction coordinate diagram for a two-step reaction like this one will feature two peaks (Draw one on the board)
	\item Elementary steps each have their own reaction rates:
	\begin{itemize}
		\item The rate law for an elementary step depends on the \emph{molecularity} of the step
		\begin{itemize}
			\item Unimolecular steps have only one reactant: \ch{AB -> A + B} (cyclobutane decomposition figure)
			\item Bimolecular steps involve an encounter between two molecules: \ch{A + B -> C} or \ch{2 A -> B} (Figure 12.17)
			\item Termolecular steps are very rare, but can occur
		\end{itemize}
		\item The rate law can be inferred from the stoichiometry of the step
    \begin{itemize}
			\item \ch{AB -> A + B} ~~ gives ~~ $rate=k\left[AB\right]$
			\item \ch{A + B -> C} ~~ gives ~~ $rate=k\left[A\right]\left[B\right]$
			\item \ch{2 A -> B} ~~ gives ~~ $rate=k\left[A\right]^2$
		\end{itemize}
		\item While each step has its own rate, the overall reaction can only proceed at the rate of the \emph{slowest} step
		\item The slowest step is therefore called the \emph{rate-limiting} step of the reaction (Figure 12.18\ldots a bit strained analogy)
		\item Looking at the reaction of \ch{NO2} and \ch{CO} above, the first step is slower so the overall reaction rate will be $rate=k\left[NO2\right]^2$
	\end{itemize}
	\item Any proposed mechanism must, at a minimum, add up to the total equation, and produce a rate law consistent with observations
	\item This is not conclusive proof of a mechanism's validity, as one could contrive infinite mechanisms within these two constraints
	\item Some elementary steps are reversible, and establish an equilibrium (subject of the next chapter):
	\begin{itemize}
		\item Consider the reaction \ch{2 NO(g) + Cl2(g) -> 2 NOCl(g)} \hspace{1em} $rate=k\left[NO\right]^2\left[Cl\right]$
		\item This observed rate law seems consistent with a single-step \emph{termolecular} mechanism, but termolecular reactions are exceptionally rare
		\item An alternative proposed mechanism is:
		
		\ch{NO(g) + Cl2(g) <=>[ $k_1$ ][ $k_{-1}$ ] NOCl2(g)} \hspace{1em} FAST
		
		\ch{NOCl2(g) + NO(g) ->[ $k_2$ ] 2 NOCl(g)} \hspace{1em} SLOW
		\item The first step will reach an equilibrium, where the forward rate will equal the reverse rate
		
		$k_1\left[\ch{NO}\right]\left[\ch{Cl2}\right] = k_{-1}\left[\ch{NOCl2}\right]$
		\item Rearrange this to give the equilibrium concentration of the intermediate $\left[\ch{NOCl2}\right]=\dfrac{k_1}{k_{-1}}\left[\ch{NO}\right]\left[\ch{Cl2}\right]$
		\item The reaction rate is ultimately determined by the formation of product in the second step:
		
		$rate=k_2\left[\ch{NOCl2}\right]\left[\ch{NO}\right]$
		\item Substitute in our expression for $\left[\ch{NOCl2}\right]$
		
		$rate=k_2\left(\dfrac{k_1}{k_{-1}}\left[\ch{NO}\right]\left[\ch{Cl2}\right]\right)\left[\ch{NO}\right]=\dfrac{k_2k_1}{k_{-1}}\left[\ch{NO}\right]^2\left[\ch{Cl2}\right]$		
	\end{itemize}
	\item Consider the reaction: \ch{2 NO(g) + O2(g) -> 2 NO2(g)} \hspace{1em} $\Delta H_{rxn}=-116.2\dfrac{kJ}{mol}$ 
	\begin{itemize}
		\item The observed rate law is: $rate=k\left[\ch{NO}\right]^2\left[\ch{O2}\right]$
		\item A proposed mechanism is:
		
		\ch{NO(g) + O2(g) <=> NO3(g)} \hspace{1em} FAST
		
		\ch{NO(g) + NO3(g) -> 2 NO2(g)} \hspace{1em} SLOW
		\item Is the reaction mechanism consistent with the observed rate law?
		\item What is the value of $k$ in terms of the elementary step rate constants?
		\item Draw a basically accurate reaction coordinate diagram for this reaction
	\end{itemize}
	\item Reaction mechanisms can become complex and interesting for certain reactions (harpoon mechanism and collisional activation)
\end{itemize}

\section{Catalysis}
\begin{itemize}
	\item Catalysts provide an alternative reaction mechanism which is faster than the uncatalyzed pathway
	\item Figure 12.19 shows a typical reaction coordinate diagram for a catalyzed reaction
	\item Catalysts appear in in the reaction mechanism, being first \emph{consumed}, then \emph{regenerated}
	\item The reaction \ch{2 H2O2(aq) -> 2 H2O(l) + O2(g)} can be catalyzed by \ch{HBr(aq)}
	
	\ch{H2O2(aq) + 2 HBr(aq) -> Br2(aq) + 2 H2O(l)} \hspace{1em} SLOW
	
	\ch{Br2(aq) + H2O2(aq) -> O2(g) + 2 HBr(aq)} \hspace{1em} FAST
	\item Note that the \ch{HBr} is consumed in the first step, but regenerated in the second step
  \item Homogeneous catalysis has the catalyst in the same phase as the reactants
  \item Heterogeneous catalysis has a catalyst in a different phase (Figure 12.23)
	\item Some ways catalysts can operate:
	\begin{itemize}
		\item Heterogeneous catalysts confine reactants to a surface, increasing the encounter frequency
		\item Enzymes hold the reactants in precise configurations, improving the steric component of the frequency factor and stabilizing the activated complex
		\item Homogeneous catalysts produce new compounds which shift electron density and weaken bonds which must be broken for the reaction to proceed
	\end{itemize}
  \item Enzymes are a special class of catalyst important for living organisms
  \begin{itemize}
    \item Enzymes are usually large proteins, sometimes incorporating metal cofactors
    \item Enzymes can be incredibly efficient by reducing the activation energy barrier to near $0$
    \item Figure 12.25 shows a schematic of how enzymes work
  \end{itemize}
	\item Analyze the following reaction mechanism for additional practice:
	
	\ch{C3H6(aq) + H^+(aq) -> C3H7^+(aq)}
	
	\ch{C3H7^+(aq) + H2O(l) -> C3H9O^+(aq)}
	
	\ch{C3H9O^+(aq) -> C3H8O(aq) + H^+(aq)}
	\begin{itemize}
		\item Give the total overall reaction
		\item Identify any catalysts and intermediates
	\end{itemize}
\end{itemize}

\paragraph*{Quiz 12.4 - Reaction Mechanisms}
\paragraph*{Homework 12.4}
\begin{itemize}
  \item 72: Rate laws of elementary steps
  \item 74: Validating a mechanism for a real reaction
  \item 78: Catalysts and mechanisms
  \item 80: Catalysis and reaction coordinate diagrams
\end{itemize}

\chapter{Fundamental Equilibrium Concepts}

\section{Chemical Equilibria}
\begin{itemize}
  \item Figure 13.1 illustrates how \ch{CO2} in our blood and blood $pH$ are mediated by equilibrium reactions
  \item Some reactions can go in both the forward and reverse directions
	
	\ch{N2O4(g) -> 2 NO2(g)} and \ch{2 NO2(g) -> N2O4(g)}
	\item Such reactions will reach an equilibrium
	\begin{itemize}
		\item Equilibrium is when the forward reaction rate and the reverse reaction rate are equal
    \item Treating these as single-step (elementary) reactions yields: $rate_f=k_f[\ch{N2O4}]$ and $rate_r=k_r[\ch{NO2}]^2$
		\item The concentrations of reactants and products remains steady indefinitely once equilibrium is reached
		\item Figure 13.2 shows how the amounts of reactant and product and the reaction rates shift over time until the rates are equalized and equilibrium is reached
		\item Equilibrium is a dynamic state -- reactions continue, they merely balance each other (Figure 13.3)
		\item The precise concentrations at equilibrium will depend on the starting conditions
	\end{itemize}
	\item The above two reactions can be combined into one equation: \ch{N2O4(g) <=> 2 NO2(g)}
  \item Figure 13.4 illustrates how vapor pressure is an equilibrium process (\ch{Br2(l) <=> Br2(g)})
\end{itemize}

\section{Equilibrium Constants}
\begin{itemize}
  \item The current concentrations of reactants and products can be summarized with the \emph{reaction quotient} $Q$
  \item For a generic reaction: \ch{m A + n B <=> x C + y D} the reaction quotient is:

    $Q_C=\dfrac{\left[C\right]^x\left[D\right]^y}{\left[A\right]^m\left[B\right]^n}$

    $Q_P=\dfrac{P_C^xP_D^y}{P_A^mP_B^n}$
  \item For heterogeneous reactions, we exclude (s) and (l) species from $Q$ and $K$ (reasons for this are a bit complex)
  \item Practice with the equation: \ch{N2(g) + 3 H2(g) <=> 2 NH3(g)} ($K_{C,300K}=2.7\times10^{8}$)
  \item As the system approaches equilibrium, the value of $Q$ approaches $K$
  \item $K$ has the same mathematical form as $Q$, but has concentrations or pressures that are already at equilibrium
  \item Both $K$ and $Q$ are unitless, for reasons we don't cover in this class
  \item At a given temperature, the equilibrium conditions may be different for different starting points, but the value of $K$ is the same
  \item The magnitude of $K$ can tell about general conditions at equilibrium
	\begin{itemize}
		\item If $K \gg 1$ then equilibrium will favor products
		\item If $K \ll 1$ then equilibrium will favor reactants
	\end{itemize}
  \item For a given starting position, we can calculate $Q$ to determine what direction the system needs to shift to reach equilibrium
    \begin{itemize}
      \item If $Q<K$ then the system must shift toward products
      \item If $Q>K$ then the system must shift toward reactants
    \end{itemize}
	\item Equilibrium can be reached whether you start with reactants or start with products (Figure 13.5)
	\item Note that \ch{N2(g)} alone or \ch{H2(g)} alone cannot lead to equilibrium from the reactant side -- both are needed
  \item We can show that $K_p = K_c\left(RT\right)^{\Delta n}$ by substituting in $P=MRT$
  \item Give the relationship between $K_p$ and $K_c$ for the formation of ammonia: $K_p = \dfrac{K_c}{\left(RT\right)^2}$
\end{itemize}

\paragraph*{Quiz 13.1 - Equilibrium Constants}
\paragraph*{Homework 13.1}
\begin{itemize}
  \item 3: Recognize when a system reaches equilibrium
  \item 9: Reactant-favored reactions
  \item 15: Reaction quotient for many reactions
\end{itemize}

\section*{Resuming Section 13.2 - Equilibrium Constants}
\begin{itemize}
  \item Using the equilibrium constant
    \begin{itemize}
      \item We can measure the concentrations at equilibrium and directly measure the equilibrium constant
      \begin{itemize}
        \item Consider the reaction: \ch{CO(g) + H2O(g) <=> CO2(g) + H2(g)}
        \item Find $K_c$ if $\left[\ch{CO}\right]=0.0600~M$, $\left[\ch{H2O}\right]=0.120~M$, $\left[\ch{C2O}\right]=0.150~M$, and $\left[\ch{H2}\right]=0.300~M$
        \item Figure 13.6 shows how many different reaction mixture compositions can all satisfy the equilibrium expression
      \end{itemize}
      \item We can use the equilibrium constant to find an unknown concentration
      \begin{itemize}
        \item \ch{CH3CO2H(aq) + H2O(l) <=> H3O^+(aq) + CH3CO2^_(aq)} \hspace{1em} $K=1.8\times10^{-5}$
        \item Find $\left[\ch{H3O^+}\right]$ if $\left[\ch{CH3COOH}\right] = 0.250~M$ and $\left[\ch{CH3COO^-}\right]=0.350~M$
      \end{itemize}
      \item We can easily find $K$ for doubled, reversed, or added equations (\emph{not in your text})
      \begin{itemize}
        \item For a reversed equation, reactants and products switch, so $K_{reverse} = K_{normal}^(-1)$
        \item For a doubled (or tripled) reaction, $K$ should be squared (or cubed)
        \item For added reactions (multi-step) $K_{total}=K_1K_2$
      \end{itemize}
      \item Find $K$ for the reaction: \ch{N2(g) + 2 O2(g) <=> 2 NO2(g)}
      
      \ch{N2(g) + O2(g) <=> 2 NO(g)} \hspace{1em} $K=2.0\times10^{-25}$
      
      \ch{2 NO(g) + O2(g) <=> 2 NO2(g)} \hspace{1em} $K=6.4\times10^{9}$
    \end{itemize}
\end{itemize}

\paragraph*{Quiz 13.2 - Working with K}
\paragraph*{Homework 13.2}
\begin{itemize}
  \item 19: Predict direction to reach equilibrium
  \item 25: Covert $K_C$ to $K_P$
  \item 52: Calculate $K$ from concentrations
\end{itemize}

\section{Shifting Equilibria: Le Ch\^atelier's Principle}
\begin{itemize}
	\item When changes are made to a system at equilibrium, it will shift in response to that change to maintain equilibrium
	\item This is called Le Ch\^atelier's Principle
	\item We can calculate $Q$ after the change and compare it to $K$
	\item In practice, though, a few simple rules are sufficient without any calculations
	\item Adding or removing a reactant or product
	\begin{itemize}
		\item If a reactant or product is added or removed, the system will respond to counteract the change
		\item A U-pipe with water is a good analogy for this shift
		\item Consider the reaction \ch{2 H2S(g) <=> 2 H2(g) + S2(g)}
		
		How would the reaction shift if each species is added or removed in turn?
	\end{itemize}
	\item Changing volume
	\begin{itemize}
		\item If the reaction volume changes, all the concentrations or pressures will change together
		\item The effect this has depends on the stoichiometry of the reaction
		\item Consider the reaction \ch{3 H2(g) + N2(g) <=> 2 NH3(g)} -- How will the reaction quotient change if the volume doubles?
		\item The shift depends on $\Delta_n$, considering only the moles of gas or aqueous substances
		\item If volume increases (dilution), the reaction will shift to the side with \emph{more} moles
		\item If volume decreases (concentration), the reaction will shift to the side with \emph{fewer} moles
		\item If $\Delta_m=0$, then the reaction is unaffected by dilution and concentration
	\end{itemize}
	\item Temperature changes
	\begin{itemize}
		\item Unlike with the other changes, a change in $T$ will actually change the value of $K$
		\item How $K$ changes depends on $\Delta H_{rxn}$
		\item We will explore this relationship mathematically later, but for now we can use a trick to determine the direction of the shift
		\item Consider ``heat'' as a reactant for endothermic reactions, and as a product for exothermic reactions
		\item Heat is not really a product or reactant (how many $g$ of heat are produced)
		\item Lowering $T$ removes heat and the reaction will respond just like removing any other reactant or product
		\item Raising $T$ adds heat and will have the same effect as adding any other reactant or product
	\end{itemize}
	\item Equilibrium is often not the most important factor in industrial settings. The Haber process is run at high temperatures to increase the reaction rate even though it pushes equilibrium toward reactants
	\item Addition of a catalyst has \emph{no} effect on the equilibrium position
\end{itemize}

\paragraph*{Quiz 13.3 - Le Ch\^atelier's Principle}
\paragraph*{Homework 13.3}
\begin{itemize}
  \item 36: How to maximize product
  \item 38: Temperature changes and equlibrium
  \item 42: Predicting response to changes
\end{itemize}

\section{Equilibrium Calculations}
\begin{itemize}
	\item If we know the equilibrium constant, we can find equilibrium concentrations based on the initial conditions
	\item We do this using an ICE table:
	\begin{itemize}
		\item I -- Initial conditions (often one or more species will have an initial concentration of $0$)
		\item C -- Change. Express the change in terms of $x$ and be sure to consider stoichiometry
		\item E -- Equilibrium conditions. These values (I + C) should be put in the equilibrium expression
		\item Once you have the equilibrium expression you can calculate the change amount ($x$)
		\item You can do this technique with pressures or molar concentrations, depending on the form of $K$
		\item If you do your change calculations in pure moles, be sure to change them into $P$ or $[]$ before you put them into the equilibrium expression
	\end{itemize}
	\item Consider the reaction \ch{H2(g) + I2(g) <=> 2 HI(g)}
	
	$5.00~mol$ \ch{H2} and $0.500~mol$ \ch{I2} are reacted in a $1.00~L$ chamber and at equilibrium $\left[\ch{HI}\right] = 0.900~M$. Find the value of $K_C$ ($324$)
	\item Consider the reaction \ch{CO(g) + H2O(g) <=> H2(g) + CO2(g)} \hspace{1em} $K_C=5.80$
	
	Find the equilibrium conditions if initial concentrations are: $\left[\ch{CO}\right]= \left[\ch{H2O}\right]=0.0125~M$
	\item Often the equilibrium expression will lead to a quadratic equation when solving for $x$
	\item The quadratic formula is: $ax^2 + bx + c = 0 \rightarrow x=\dfrac{-b\pm\sqrt{b^2-4ac}}{2a}$
	\item Consider the reaction: \ch{I2(g) <=> 2 I(g)} \hspace{1em} $K_{P,1000K}=0.260$
	
	Find the equilibrium conditions if a reaction chamber is initially charged with $0.200~atm$ \ch{I} and $0.00500~atm$ \ch{I2}
	\item Some quadratic equations can be greatly simplified by recognizing when $x$ is small compared to initial amounts
	\begin{itemize}
		\item If $x$ is small, then it can be neglected from any species with an initial amount
		\item First solve the equation \emph{assuming} that $x$ can be neglected
		\item Compare the solved value of $x$ to the amounts it was neglected from
		\item If $x$ is less than $5\%$ of those values then your simplification was valid
		\item If not, then you must go back and solve the complete quadratic equation
	\end{itemize}
	\item Consider the reaction \ch{HCOOH(aq) + H2O(l) <=> H3O^+(aq) + HCOO^-(aq)} \hspace{1em} $K_C=1.8\times10^{-4}$
	
	Find the equilibrium conditions for a solution that begins with $\left[\ch{HCOOH}\right]=0.250~M$
\end{itemize}

\paragraph*{Quiz 13.4 - ICE Tables}
\paragraph*{Homework 13.4}
\begin{itemize}
  \item 54: Calculate $K$ using an ICE table
  \item 76: Use ICE table to calculate equlibrium amounts
  \item 88: Calculate pressures using an ICE table
\end{itemize}

\chapter{Acid-Base Equilibria}

\section{Brønsted-Lowry Acids and Bases}
\begin{itemize}
  \item Arrhenius acid/base theory defines acids and bases in terms of \ch{H3O^+} and \ch{OH^-} ions
	\begin{itemize}
		\item \ch{HCl(aq) + H2O(l) -> H3O^+(aq) + OH^-(aq)}
		\item \ch{NaOH(s) ->[water] Na^+(aq) + OH^-(aq)}
	\end{itemize}
	\item \ch{H^+} vs \ch{H3O^+}
	\begin{itemize}
		\item In the past, you may have used \ch{H^+(aq)} in your equations
		\item I will tend to use \ch{H3O^+} instead, but either way is acceptable
		\item Really, \ch{H^+} is a flagrant lie. Bare protons don't exist in water. \ch{H3O^+} is also a little bit of a lie. The extra proton and the charge create clusters of many water molecules. It gets really complicated
	\end{itemize}
  \item Arrhenius theory is unable to account for reactions that don't involve \ch{H3O^+} and \ch{OH^-} ions directly
	\item Consider \ch{HNO2(aq) + ClO2^-(aq) <=> HClO2(aq) + NO2^-(aq)} -- should this reaction be considered acid/base?
	\item Br\o nsted-Lowry theory defines acids as proton \emph{donors} and bases as proton \emph{acceptors}
	\item \ch{HNO2} is the acid, and \ch{ClO2} is the base
	\item Conjugate pairs
	\begin{itemize}
		\item We can also consider the reverse reaction, where \ch{HClO2} is the acid and \ch{NO2^-} is the base
		\item We call these linked species \emph{conjugate pairs}
		\item \ch{HNO2} is the conjugate acid to \ch{NO2^-}, while \ch{NO2^-} is the conjugate base to \ch{HNO2}
		\item The strengths of a conjugate pair are inverse to each other -- A stronger acid has a weaker conjugate base and vice-versa
	\end{itemize}
	\item Some acids are \emph{multiprotic} and some bases are \emph{multibasic}
	\begin{itemize}
		\item Consider the series \ch{H2SO3 <=> HSO3^- <=> SO3^{2-}} 
		\item Or the series \ch{H3PO4 <=> H2PO4^- <=> HPO4^{2-} <=> PO4^{3-}}
		\item \ch{H2SO3} is a diprotic acid, and \ch{SO3^{2-}} is a dibasic base because they can exchange 2 protons
		\item \ch{H3PO4} is a triprotic acid, and \ch{PO4^{3-}} is a tribasic base because they can exchange 3 protons
		\item The intermediates, \ch{HSO3^-}, \ch{ H2PO4^-} and \ch{HPO4^{2-}} can act as either an acid or a base
		\item These types of ions are called \emph{amphoteric} or \emph{amphiprotic}
		\item Whether they act as an acid or a base depends on the context -- what is their reaction partner
	\end{itemize}
\end{itemize}

\section{pH and pOH}
\begin{itemize}
  \item Autoionization of Water
    \begin{itemize}
      \item Water is also amphoteric: \ch{H3O^+ <=> H2O <=> OH^-}
      \item Because of this, water will undergo autoionization: \ch{2 H2O(l) <=> H3O^+(aq) + OH^-(aq)} \hspace{1em} $K_w=1.00\times10^{-14}$
      \item For pure water, this leads to concentrations of $\left[\ch{H3O^+}\right]=\left[\ch{OH^-}\right]=1.00\times10^{-7}$
      \item This water ionization equilibrium is the great arbiter of acid/base chemistry. It defines what is an acid, what is a base, and what are their various strengths
      \item Even for reactions which don't explicitly contain water (\ch{NH3(aq) + HNO2(aq) -> NH4^+(aq) + NO2^-{aq}}), water is actually mediating the proton exchange behind the scenes
      \item Because of this, acidity and basicity are defined by the balance between $\left[\ch{H3O^+}\right]$ and $\left[\ch{OH^-}\right]$ (Table 14.1)
      \begin{itemize}
        \item $\left[\ch{H3O^+}\right] > \left[\ch{OH^-}\right]$ is an acid
        \item $\left[\ch{H3O^+}\right] < \left[\ch{OH^-}\right]$ is a base
        \item $\left[\ch{H3O^+}\right] = \left[\ch{OH^-}\right]$ is neutral
      \end{itemize}
      \item We can always find $\left[\ch{H3O^+}\right]$ or $\left[\ch{OH^-}\right]$ from the other, based on $K_w = \left[\ch{H3O^+}\right]\left[\ch{OH^-}\right]$
    \end{itemize}
  \item Acidity of a solution is summarized by the quantity $pH = -\log\left[\ch{H3O^+}\right]$
	\item Neutral solutions have $pH=7$, acidic solutions have $pH<7$, and basic solutions have $pH>7$
	\item $0$ and $14$ are actually not boundaries at all, and you \emph{can} have solutions with $pH<0$ or $pH>14$
	\item We can find the $\left[\ch{H3O^+}\right]$ by $\left[\ch{H3O^+}\right]=10^{-pH}$
	\item Table 14.2 shows both the $pH$ and $\left[\ch{H3O^+}\right]$ for several common substances
	\item We can make similar calculations for $\left[\ch{OH^-}\right]$ and $pOH$
	\item This gives us the interesting relationship that $pH+pOH=14$ -- draw the conversion rectangle
	\item We can measure $pH$ in several ways:
	\begin{itemize}
		\item Color indicators are chemicals which change color based on $pH$ conditions
		\item We will see that these indicators are themselves weak acids/base conjugate pairs
		\item Indicators can be dissolved in the solution or applied onto paper strips
		\item By mixing several indicators, we can get a different value at each $pH$, making a ``universal indicator''
		\item Figure 14.5 shows how a universal indicator $pH$ paper works
		\item We can also measure an electrochemical response using a $pH$ probe (Figure 14.4)
		\item How these $pH$ probes work is a bit complicated, but they can measure $\left[\ch{H3O^+}\right]$ across a wide range
	\end{itemize}
\end{itemize}

\section{Relative Strengths of Acids and Bases}
\begin{itemize}
	\item Acids and bases can be \emph{strong} or \emph{weak}
	\begin{itemize}
		\item Strong acids and bases are those which dissociate completely (or, at least, nearly so) 
		\item Weak acids and bases establish an equilibrium which is usually highly reactant-favored
    \item Figure 14.6 lists some strong acids and bases
	\end{itemize}
  \item Weak acids and weak bases will react with water to reach an equilibrium
	\item Because equilibrium concentrations are different from initial concentrations, pedantic people (like me!) will sometimes use \emph{formal} concentration, $F$, instead of molar concentration
	\item The equilibrium constant for their hydrolysis reactions are given the name $K_a$ for acids, and $K_b$ for bases
	\item Figure 14.8 shows the $K_a$ and $K_b$ values for a number of different acids and bases
	\item It is also sometimes useful to find the $pK_a = -\log K_a$ or the $pK_b=-\log K_b$
	\item We can use these equilibrium constants with an ICE table to find the $pH$ of a solution under different circumstances
	\begin{itemize}
		\item Weak acids always follow the same format: $K_a=\dfrac{\left[\ch{A^-}\right]\left[\ch{H3O^+}\right]}{\left[\ch{HA}\right]}$
		\item Weak bases always follow the same format: $K_b=\dfrac{\left[\ch{HB^+}\right]\left[\ch{OH^-}\right]}{\left[\ch{B}\right]}$
		\item We will often be able to use the simplification that the change is much less than the initial amounts
	\end{itemize}
	\item Find the $pH$ for a $0.250~F$ solution of nitrous acid ($K_a=4.0\times10^{-4}$)
	\item Find the $pH$ for a $0.325~F$ solution of pyridine ($K_b=1.7\times10^{-9}$)
	\item A few useful relations that are not in your textbook:
	\begin{itemize}
		\item For conjugate acid/base pairs: $K_aK_b=K_w$
		\item For the reaction between an acid and a base: $K=\dfrac{K_aK_b}{K_w}$
	\end{itemize}
  \item Relating Acid Strength to Structure
  \begin{itemize}
    \item Acid strength ultimately comes from the strength of the \ch{H} bond and the stability of the product ions
    \item Bond strength:
    \begin{itemize}
      \item Weaker \ch{H} bonds lead to stronger acids
      \item Longer bonds tend to be weaker, hence the strength of \ch{HI} is greater than the strength of \ch{HF} (Figure 14.11)
      \item Electronegative groups nearby pull electrons away from the \ch{H} bond and make it weaker (Figure 14.12)
    \end{itemize}
    \item Ion stability:
    \begin{itemize}
      \item Even strong \ch{H} bonds can be acidic if the anion after deprotonation is particularly stable
      \item Consider the structure of acetic acid and acetate
      \begin{itemize}
        \item Acetic acid does not exhibit resonance, but acetate ion does
        \item By losing a hydrogen, the acetate ion can stabilize with resonance
        \item This makes acetic acid stronger than we might have assumed based only on the \ch{O-H} bond strength
      \end{itemize}
    \end{itemize}
  \end{itemize}
\end{itemize}

\section{Hydrolysis of Salts}
\begin{itemize}
  \item Some salts have no effect on $pH$ when dissolved on water, while others do
	\item To determine the acid/base strength of a salt, look at the individual ions which dissociate in the water
	\item Cations are usually neutral with two exceptions:
	\begin{itemize}
		\item \ch{NH4^+} and others based off of it are weak acids
		\item Some metal cations can act as acids -- we will cover this more in the Lewis Acids section
	\end{itemize}
	\item The anions are usually where the activity lies:
	\begin{itemize}
		\item Many anions are neutral, such as the conjugates to strong acids
		\item Some anions are amphoteric, and their effect depends on their $K_a$ and $K_b$ values
		\item If $K_a>K_b$, then the anion will be acidic, and if $K_a<K_b$, then the ion will be basic
		\item Some anions are simply weak bases
	\end{itemize}
	\item You can then find the $pH$ based on the acid/base activity of the individual ions
	\item Find the $pH$ of a $0.125~M$ solution of \ch{CaF2}
  \item Some metal cations are acidic in a surprising way
    \begin{itemize}
      \item The metal cation will complex one or more water molecules
      \item The bound water molecules have weaker O-H bond strengths, so one \ch{H^+} is donated
      \item Figure 14.13 shows this process for \ch{Al^{3+}}
      \item The text lists the $pK_a$ for several metal cations
      \item This type of acid is sometimes called a \emph{Lewis} acid
    \end{itemize}
\end{itemize}

\section{Polyprotic Acids}
\begin{itemize}
  \item For polyprotic acids and bases, each step has its own $K_a$ or $K_b$ values
	\item Each successive proton loss will have significantly lower acid strength
	\item Table  shows $K_a$ values for several polyprotic acids
	\item Amphoteric species can establish multiple simultaneous and interdependent equilibria, but we will only look at the simple problems
	\item Find pH, $\left[\ch{H2C6H6O6}\right]$, $\left[\ch{HC6H6O6^-}\right]$, and  $\left[\ch{C6H6O6^{2-}}\right]$ for a $0.500~F$ solution of Ascorbic acid
	\begin{itemize}
		\item First, solve the ICE table for the first deprotonation
		\item Then, use the $\left[\ch{H3O^+}\right]$ and $\left[\ch{HC6H6O6^-}\right]$ as starting values for the second deprotonation
		\item Because the $\left[\ch{H3O^+}\right]$ will not change much, we don't have to revisit the first equilibrium
	\end{itemize}
\end{itemize}

\section{Buffers}
\begin{itemize}
  \item Introduction to Buffer Solutions
  \begin{itemize}
    \item Buffer solutions resist a change in $pH$ when acid or base is added to them
    \item Buffer solutions are everywhere in nature -- blood, soil, ocean water, etc.
    \item The common-ion effect is critical to how buffer solutions work
    \begin{itemize}
      \item Consider the equilibrium reaction: \ch{HNO2(aq) + H2O(l) <=> NO2^-(aq) + H3O^+(aq)}
      \item Adding \ch{NaNO2} will affect this equilibrium because it adds nitrite ion to solution
      \item Le Ch\^atelier's principle will shift the reaction left, reducing the effect \ch{HNO2} has on $pH$
      \item Any equilibrium reaction involving ions can be affected by addition of other salts containing those emails -- This is the common ion effect
    \end{itemize}
    \item Buffer solutions contain amounts of both members of a weak conjugate acid/base pair
    \begin{itemize}
      \item Adding strong acid or strong base to the buffer will react with the weak base/acid rather than with water, resulting in a suppressed change in $pH$
      \item Figure 14.14 shows how $pH$ changes after adding $1ml$ of $0.01M$ \ch{HCl} in a buffered and unbuffered solution starting at $pH=8.00$ 
      \item The buffer can only absorb a certain amount of acid or base, called the buffer capacity, which will be covered later
      \item Figure 14.15 shows how concentrations of the buffering species change when acid or base are added
    \end{itemize}
  \end{itemize}
  \item The Henderson-Hasselbach Equation
  \begin{itemize}
    \item For a buffer solution, we can solve an ICE table with initial amounts of both \ch{HA} and \ch{A^-}
    \item In this case, the equilibrium expression simplifies to: $\left[\ch{H3O^+}\right]=K_a\dfrac{\left[\ch{HA}\right]}{\left[\ch{A^-}\right]}$
    \item We can calculate the $pH$ from here to be: $pH = pK_a-\log\dfrac{\left[\ch{HA}\right]}{\left[\ch{A^-}\right]}$
    \item This is the Henderson-Hasselbach equation
    \item Buffer capacity describes the range of $pH$ values where a buffer works
    \begin{itemize}
      \item The buffer performs best (smallest $pH$ changes) when $pH=pK_a$
      \item $pH$ changes become more pronounced as more acid or base is added
      \item Once the $pH$ strays beyond $\pm 1$ of $K_a$, the buffer capacity is exceeded and the buffer will stop working
      \item Higher concentrations of \ch{HA} and \ch{A^-} can absorb more acid or base before its capacity is exceeded
    \end{itemize}
    \item \ch{H2S} has $K_a=9.1\times10^{-8}$. Find the $pH$ of a solution if $2.7~g$ of \ch{H2S} and $1.5~g$ of \ch{NaHS} are dissolved in $0.50~L$ of water
    \item \ch{H2PO4^-} has $pK_a=6.8$ and is responsible for regulating the $pH$ of blood at $pH=7.4$. 
    \begin{itemize}
      \item If $\left[\ch{HPO4^{2-}}\right]\approx0.200~M$, find the $\left[\ch{H2PO4^{-}}\right]$ to maintain proper blood $pH$
      \item How many $g$ of \ch{HCl} could be added to $5.0~L$ of blood before the buffer capacity is exceeded?
      \item How many $g$ of \ch{NaOH} could be added to $5.0~L$ of blood before the buffer capacity is exceeded?	
    \end{itemize}
  \end{itemize}
\end{itemize}

\section{Acid-Base Titrations}
\begin{itemize}
  \item A titration is the gold standard technique for determining the concentration of a solute
	\item The unknown substance is called the \emph{analyte}
	\item A solution of suitable reaction partner, called the \emph{titrant} is slowly added until the analyte is completely consumed
	\item Acid/base, redox, and precipitation reactions can all be titrated, though the first two are by far more common
	\item Figures 17.3 and 17.4 show typical titration setups, with different methods of determining the exact point where the analyte is consumed (called the \emph{equivalence point})
	\item Figure 14.18(a) shows a titration curve for a strong acid with a strong base
	\begin{itemize}
		\item Point A will be the pH of the analyte solution, $pH=-\log\left[\ch{H3O^+}\right]=-\log\left[\ch{HA}\right]$
		\item In region B the base is reacting with the acid and causing the $pH$ to rise
		\item Solve the $pH$ using a BCA table
		\item At point C, the equivalence point, the base has precisely neutralized the acid, and $pH=7.00$ exactly
		\item In region D, the $pH$ changes as excess base is added
	\end{itemize}
	\item At the equivalence point, moles of titrant added is equal to moles of analyte
	\item This relation can be summarized as $M_aV^\circ_a=M_tV_{eq}$
  \item Titrations of Weak Acids and Weak Bases
  \begin{itemize}
    \item Titrating a weak acid or base follows a similar process as titrating a strong acid or base
    \item The buffering property of weak acids and bases does change the details of the titration curve (Figure 14.18(b))
    \begin{itemize}
      \item The starting point A must be solved using a ICE table after the concentration is known
      \item Region B is called the buffer region, and will be centered around $K_a$
      \item Solve for the $pH$ using the Henderson-Hasselbach equation
      \item $pH=pK_a$ at precisely half of the equivalence volume
      \item Point C, the equivalence point, will have a pH determined by the strength of the conjugate base (Solve using an ICE table)
      \item In region D, the $pH$ is governed by the excess base added as in strong acid/base titrations
      \item These curves can be very different for different acids, even at the same concentration
      \item Note the strong inflection in curves for very weak acids
    \end{itemize}
    \item The equivalence point is recognized as the point with steepest $pH$ change	
    \item Multiprotic acids or bases will pass through more than one equivalence point (Draw titration curve)
  \end{itemize}
  \item Indicators in Acid-Base Titrations
  \begin{itemize}
    \item Color indicators are often used in titrations instead of $pH$ probes
    \item Color indicators are themselves weak acid/base conjugate pairs
    \begin{itemize}
      \item \ch{HA(aq) + H2O(l) <=> A-(aq) + H3O^+(aq)}
      \item \ch{HA} is one color, and \ch{A^-} is another
      \item The indicator is in such low concentration, that the $pH$ is governed by the titrant and analyte rather than by the indicator
      \item The ratio of \ch{HA} and \ch{A^-} responds to the $pH$ according to the H-H equation
      \item Over the buffer range ($pK_a \pm1$) the indicator goes from primarily \ch{HA} to primarily \ch{A^-} and the solution color changes
    \end{itemize}
    \item A color indicator should be carefully chosen to match the titration it will be used for
    \begin{itemize}
      \item The expected equivalence point $pH$ should lie within the buffer range for the indicator
      \item The point of color change, when you actually stop the titration, is called the \emph{end point} and should be close to the true equivalence point
      \item Figure 14.20 shows how the end point and equivalence point are situated in a titration curve
      \item Figure 14.19 shows the appropriate ranges for some common indicators
    \end{itemize}
  \end{itemize}
\end{itemize}

\chapter{Equilibria of Other Reaction Classes}

\section{Precipitation and Dissolution}
% \begin{itemize}
%   \item 
% \end{itemize}

\section{Lewis Acids and Bases}
% \begin{itemize}
%   \item 
% \end{itemize}

\section{Coupled Equilibria}
% \begin{itemize}
%   \item 
% \end{itemize}

\chapter{Thermodynamics}

\section{Spontaneity}
% \begin{itemize}
%   \item 
% \end{itemize}

\section{Entropy}
% \begin{itemize}
%   \item 
% \end{itemize}

\section{The Second and Third Laws of Thermodynamics}
% \begin{itemize}
%   \item 
% \end{itemize}

\section{Free Energy}
% \begin{itemize}
%   \item 
% \end{itemize}

\chapter{Electrochemistry}

\section{Review of Redox Chemistry}
% \begin{itemize}
%   \item 
% \end{itemize}

\section{Galvanic Cells}
% \begin{itemize}
%   \item 
% \end{itemize}

\section{Electrode and Cell Potentials}
% \begin{itemize}
%   \item 
% \end{itemize}

\section{Potential, Fee Energy, and Equilibrium}
% \begin{itemize}
%   \item 
% \end{itemize}

\section{Batteries and Fuel Cells}
% \begin{itemize}
%   \item 
% \end{itemize}

\section{Corrosion}
% \begin{itemize}
%   \item 
% \end{itemize}

\section{Electrolysis}
% \begin{itemize}
%   \item 
% \end{itemize}

\setcounter{chapter}{20}
\chapter{Nuclear Chemistry}

\section{Nuclear Structure and Stability}
% \begin{itemize}
%   \item 
% \end{itemize}

\section{Nuclear Equations}
% \begin{itemize}
%   \item 
% \end{itemize}

\section{Radioactive Decay}
% \begin{itemize}
%   \item 
% \end{itemize}

\section{Transmutation and Nuclear Energy}
% \begin{itemize}
%   \item 
% \end{itemize}

\section{Uses of Radioisotopes}
% \begin{itemize}
%   \item 
% \end{itemize}

\section{Biological Effects of Radiation}
% \begin{itemize}
%   \item 
% \end{itemize}

\backmatter
\chapter{Errata}
% \begin{itemize}
%   \item 
% \end{itemize}

\end{document}
