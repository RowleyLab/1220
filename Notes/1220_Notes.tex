\documentclass[12pt, openany, letterpaper]{memoir}
\usepackage{NotesStyle}
%\renewcommand\thesection{\thechapter\Alph{section}}
%\renewcommand\thesubsection{\thesection.\Numeral{subsection}}

\begin{document}
\title{CHEM 1220 Lecture Notes\\ OpenStax Chemistry 2e}
\author{Matthew Rowley}
\date{\today}
\mainmatter
\maketitle
\chapter*{Course Administrative Details}
\begin{itemize}
	\item My office hours
	\item Intro to my research
	\item Introductory Quiz
	\item Grading details
	      \begin{itemize}
		      \item Exams - 40, Final - 15, Online Homework - 15, Book Homework - 15, Quizzes - 15
		      \item Online homework
		      \item Frequent quizzes
	      \end{itemize}
	\item Importance of reading and learning on your own
	\item Learning resources
	      \begin{itemize}
		      \item My Office Hours
		      \item Tutoring services - \href{https://www.suu.edu/academicsuccess/tutoring/}{https://www.suu.edu/academicsuccess/tutoring/}
	      \end{itemize}
	\item Show how to access Canvas
	      \begin{itemize}
		      \item Calendar, Grades, Modules, etc.
		      \item Quizzes
		      \item Textbook
	      \end{itemize}
	\item Introduction to chemistry
	      \begin{itemize}
		      \item Ruby fluorescence
		      \item Levomethamphetamine
		      \item Rubber band elasticity
		      \item Structure of the periodic table
		      \item Salt on ice and purifying hydrogen peroxide
	      \end{itemize}
\end{itemize}

\setcounter{chapter}{-1}
\chapter{1210 Review}

There is a whole semester of material from 1210, and these are only the topics which are \emph{most} important for success in 1220

\begin{itemize}
	\item Composition of atoms and ions (protons, neutrons and electrons)
	\item Chemical formulas and names
	\begin{itemize}
		\item Formulas and molar masses
		\item Polyatomic ion names
		\item Naming ionic compounds
		\item Naming binary molecular compounds
		\item Naming acids
	\end{itemize}
	\item Balancing molecular equations
	\item Solubility rules
	\item Fundamentals of acid/base chemistry
	\item Measurements in chemistry
	\begin{itemize}
		\item Converting from measurements to moles and back
		\item Stoichiometry and predicting amounts
		\item Limiting reactants
	\end{itemize}
	\item Enthalpy of reaction and heat equations
	\item Lewis structures
\end{itemize}
\paragraph*{CHEM 1210 Review Quiz}

\setcounter{chapter}{9}
\chapter{Liquids and Solids}

\section{Intermolecular Forces}
\begin{itemize}
  \item Many physical properties of solids, liquids, and gases can be explained by the strength of attractive forces between particles (Figure 10.5)
  \item Phase changes happen due to the interplay between kinetic energy and intermolecular forces (Figure 10.2)
  \item Pressure can also play a role in phase changes, as discussed later
  \item These \emph{intermolecular forces} come in different varieties
  \begin{itemize}
    \item Dispersion Forces Non-polar molecules, impacted by polarizability, molecular weight, and surface area
    \begin{itemize}
      \item Dominant in non-polar molecules
      \item Created by induced dipoles (Figure 10.6)
      \item Impacted by polarizability (Table 10.1)
      \item Impacted by molecular weight (hydrocarbons from methane to wax)
      \item Impacted by molecule shape (Figure 10.7 compares the boiling points of pentane isomers)
    \end{itemize}
    \item Dipole-Dipole Forces
    \begin{itemize}
      \item Dominant in polar molecules
      \item Results from attraction between permanent dipoles (Figure 10.9)
    \end{itemize}
  \item Hydrogen Bonding
    \begin{itemize}
      \item Dominant only in molecules capable of hydrogen bonding
      \item Must contain a hydrogen-donor atom (H attached to N, O, or F)
      \item Must contain a hydrogen-acceptor atom (lone pair of electrons attached to N, O, or F)
      \item Hydrogen bonds are more than just particularly strong dipole-dipole forces. They have strong directionality according to VSEPR
      \item Figures 10.10, 10.14, and other figures on the Internet show water, DNA, and proteins all organized by hydrogen bonds
      \item Figures 10.11 and 10.12 illustrate how much hydrogen bonds exceed dipole-dipole forces in strength
    \end{itemize}
  \end{itemize}
\end{itemize}

\section{Properties of Liquids}
\begin{itemize}
  \item Viscosity is a fluid's resistance to flow
  \begin{itemize}
    \item We intuitively know that both water and honey flow\ldots but at very different rates
    \item Viscosity is proportional to the strength of intermolecular forces (high IF = high viscosity)
    \item As temperature increases, kinetic energy is able to overcome intermolecular forces and viscosity decreases
    \item Table 10.2 gives the viscosities of some common substances (note the unusual units!)
  \end{itemize}
  \item Surface tension is a force which minimizes a fluid's surface area
  \begin{itemize}
    \item Cohesive vs. adhesive forces
    \item Bulk molecules have lower energy than surface molecules due to being \emph{surrounded} by cohesive forces (Figure 10.16)
    \item Figure 10.17 illustrates a waterbug supporting itself on water surface tension
    \item Surface tension is often in conflict with gravity and other forces, making most liquids rounded but not perfect spheres
    \item Surface tension is proportional to intermolecular forces (Table 10.3)
    \item Surface tension can be strongly affected by addition of certain solutes, called surfactants
  \end{itemize}
  \item Capillary action is a force between a fluid and narrow channels or capillaries of solid materials
  \begin{itemize}
    \item Due to adhesive forces with the solid, liquids will be drawn up (or, less often, pushed down) a capillary
    \item Figure 10.19 shows how paper towels are made to maximize capillary action, so they soak up water-based spills
    \item The top of the liquid (called the meniscus) will curve differently depending on the reletive strength of cohesive and adhesive forces (Figure 10.18)
    \item Figure 10.20 shows capillary action in a variety of situations, including capillary repulsion
    \item Remember that when measuring volumes, convention is to read the \emph{bottom} of the meniscus regardless of how it curves
    \item Don't worry about the formula given here
  \end{itemize}
\end{itemize}
\paragraph*{Quiz 10.1 - Intermolecular Forces and Liquid Properties}
\paragraph*{Homework 10.1}
\begin{itemize}
  \item 10.11: Predicting trends in boiling points
  \item 10.21: Identifying intermolecular forces
  \item 10.25: Affect of temperature on viscosity
\end{itemize}

\section{Phase Transitions}
\begin{itemize}
  \item Vaporization and condensation are the transitions between liquid and gas phases
  \begin{itemize}
    \item The enthalpy of vaporization $\left(\Delta H_{vap}\right)$ is the energy required to transition from liquid to gas phase
    \item Enthalpy of condensation is the opposite $\Delta H_{con} = - \Delta H_{vap}$
    \item In a closed volume, these processes will reach a \emph{dynamic equilibrium}
    \item The partial pressure of the liquid at this equilibrium state is called its \emph{vapor pressure} (Figure 10.22)
    \item Higher intermolecular forces lead to lower vapor pressures
    \item Higher temperatures increase the vapor pressure due to increased kinetic energy (Figure 10.23)
  \end{itemize}
  \item Boiling points
  \begin{itemize}
    \item Figure 10.24 shows vapor pressure curves and the normal boiling points of several liquids
    \item Boiling points generally depend on the pressure (pressure cookers, boiling water to freezing, etc.)
    \item The Clausius-Clapeyron equation defines these curves (Note the rearrangments I've made)
      \\ $P=Ae^{\nicefrac{-\Delta H_{vap}}{RT}}$ \hspace{2em} 
      $\ln P = -\dfrac{\Delta H_{vap}}{RT}+\ln A$ \hspace{2em}
      $\ln\left(\dfrac{P_2}{P_1}\right)=-\dfrac{\Delta H_{vap}}{R}\left(\dfrac{1}{T_2}-\dfrac{1}{T_1}\right)$ 
  \end{itemize}
  \item Fusion (melting), freezing, sublimation, and deposition all have their enthalpies and transition temperatures
  \item These enthalpies are state functions, such that $\Delta H_{sub} = \Delta H_{fus}+\Delta H_{vap}$ (Figure 10.28)
  \item Heating and Cooling curves
  \begin{itemize}
    \item When heat is added to a system, it will either cause a phase change, or a change in temperature
    \item For phase changes, $q=n\Delta H_{change}$
    \item For temperature changes, $q=mc\Delta T$, where $c$ is the specific heat for that substance and phase
    \item Sometimes $\Delta H_{change}$ is given as a -per gram value, and sometimes $c$ is given as a -per mole value, but usually not :(
    \item Figure 10.29 shows a typical heating curve (Work example 10.10 in the text)
  \end{itemize}
\end{itemize}
\paragraph*{Quiz 10.2 - Heating Curves}
\paragraph*{Homework 10.2}
\begin{itemize}
  \item 10.31: Temperature during a phase transition
  \item 10.39: Definition of normal boiling point
  \item 10.51: Heating curve problem
\end{itemize}

\section{Phase Diagrams}
\begin{itemize}
  \item The stable phase at different temperatures and pressures is best illustrated with a phase diagram (Figures 10.30, 10.31)
  \item We can tell at a glance what transitions might occur as we increase or decrease either the temperature or pressure
  \item Note that at some pressures, sublimation may occur instead of fusion
  \item The triple point is a unique point where liquid, solid, and gas can all exist at equilibrium (contrast with a glass of icy water on a humid day)
  \item The critical point is where the distinction between liquid and solid phases disappears
  \item Figure 10.34 shows the phase diagram of \ch{CO2}
  \item Supercritical fluids exhibit some interesting properties, and are often great solvents (Nile Blue Youtube video)
  \item Critical points vary widely depending on the intermolecular forces, and other factors (Table in text)
\end{itemize}
\paragraph*{Quiz 10.3 - Phase Diagrams}

\section{The Solid State of Matter}
\begin{itemize}
  \item Solids can be divided into \emph{crystalline} and \emph{amorphous} based on their structure at atomic scales
  \item Figure 10.37 shows the difference generally, Figure 10.38 shows crystalline and amorphous \ch{SiO2}
  \item Amorphous solids will not exhibit a sharp fusion transition temperature, but will instead grow soft and maleable over a temperature range
  \item Crystalline solids are diverse but always show long-range repeating order in their structure
  \begin{itemize}
    \item Ionic solids (Figure 10.39) have high melting points, cleave along planes, and conduct electricity only in the liquid phase
    \item Metallic solids (Figure 10.40) have mostly high melting points, are maleable and ductile, and conduct electricty and heat well
    \item Covalent network solids (Figure 10.41) have very high melting points and are electrical insulators
    \item Molecular solids (Figure 10.42) Have low to very low melting points and are electrical insulators
    \item Crystalline solid properties are summarized in Table 10.4
  \end{itemize}
  \item Even crystalline solids do not have perfect structure. Various types of defects are illustrated in Figure 10.45
\end{itemize}

\section{Lattice Structures in Crystalline Solids}
\begin{itemize}
  \item The structure of a crystalline solid is represented by a \emph{unit cell}, the smallest repeatable unit of the structure
  \item Sometimes this microscopic structure is evidently manifested on macroscopic scales, but sometimes it isn't
  \item Unit cells are defined by lattice points that often lie at the center of certain atoms, and the cell edges often cut atoms in half, quarter, etc.
  \item Unit cells of metals
  \begin{itemize}
    \item For metals, we should keep track of the quantity of atoms in a unit cell, the coordination number, and the relationship between the atomic radius and unit cell edge length
    \item Simple cubic (Figure 10.49) 1 atom, Coordination=6, $l=2r$
    \item Body-centered cubic (Figure 10.51) 2 atoms, Coordination=8, $l=\frac{4}{\sqrt{3}}r$
    \item Face-centered cubic (Figure 10.52) 4 atoms, Coordination=12, $l=\sqrt{8}r$
    \item Figure 10.54 shows hexagonal closest packed and cubic closest packed structures
    \item Find the radius of a gold atom, which has fcc structure and a density of $19.283\nicefrac{g}{cm^3}$ ($136pm$)
    \item Find the density of polonium, which has sc structure and an atomic radius of $140pm$ ($9.20\nicefrac{g}{cm^3}$)
    \item Figure 10.56 shows many non-cubic structures which are common as well
  \end{itemize}
  \item Unit cells of ionic compounds
  \begin{itemize}
    \item Anions are generally larger than cations, so ionic lattice points are generally the centers of anions
    \item Cations occupy holes in the anionic lattice (Figures 10.57 and 10.58)
    \item Unit cells of ionic structures share names with the metallic cells but look different because of the cations
    \item Simple cubic (Figure 10.59)
    \item Face-centered cubic (rock salt structure) (Figure 10.60)
    \item Zinc blende (Figure 10.61)
    \item Find the ionic bond length for \ch{NaCl} which has rock salt structure and denisty of $2.17\nicefrac{g}{cm^3}$ ($l=564pm$)
  \end{itemize}
  \item Crystal structure is determined through X-ray crystallography
  \begin{itemize}
    \item X-rays reflected off a crystal surface can combine destructively or constructively to produce an interference pattern (Figure 10.63)
    \item The X-rays will take different pathlengths depending on the angle of the X-ray beam and the crystal lattice constant (Figure 1.64)
    \item An experimental setup and actual diffractogram are shown in Figures 10.65 and 10.66
    \item We have a powerful X-ray instrument here at SUU
  \end{itemize}
\end{itemize}

\chapter{Solutions and Colloids}

\section{The Dissolution Process}
% \begin{itemize}
%   \item 
% \end{itemize}

\section{Electrolytes}
% \begin{itemize}
%   \item 
% \end{itemize}

\section{Solubilty}
% \begin{itemize}
%   \item 
% \end{itemize}

\section{Colligative Properties}
% \begin{itemize}
%   \item 
% \end{itemize}

\section{Colloids}
% \begin{itemize}
%   \item 
% \end{itemize}

\chapter{Kinetics}

\section{Chemical Reaction Rates}
% \begin{itemize}
%   \item 
% \end{itemize}

\section{Factors Affecting Reaction Rates}
% \begin{itemize}
%   \item 
% \end{itemize}

\section{Rate Laws}
% \begin{itemize}
%   \item 
% \end{itemize}

\section{Integrated Rate Laws}

\section{Collision Theory}

\section{Reaction Mechanisms}

\section{Catalysis}

\chapter{Fundamental Equilibrium Concepts}

\section{Chemical Equilibria}

\section{Equilibrium Constants}

\section{Shifting Equilibria: Le Ch\^atelier's Principle}

\section{Equilibrium Calculations}

\chapter{Acid-Base Equilibria}

\section{Brønsted-Lowry Acids and Bases}

\section{pH and pOH}

\section{Relative Strengths of Acids and Bases}

\section{Hydrolysis of Salts}

\section{Polyprotic Acids}

\section{Buffers}

\section{Acid-Base Titrations}

\chapter{Equilibria of Other Reaction Classes}

\section{Precipitation and Dissolution}

\section{Lewis Acids and Bases}

\section{Coupled Equilibria}

\chapter{Thermodynamics}

\section{Spontaneity}

\section{Entropy}

\section{The Second and Third Laws of Thermodynamics}

\section{Free Energy}

\chapter{Electrochemistry}

\section{Review of Redox Chemistry}

\section{Galvanic Cells}

\section{Electrode and Cell Potentials}

\section{Potential, Fee Energy, and Equilibrium}

\section{Batteries and Fuel Cells}

\section{Corrosion}

\section{Electrolysis}

\setcounter{chapter}{20}
\chapter{Nuclear Chemistry}

\section{Nuclear Structure and Stability}

\section{Nuclear Equations}

\section{Radioactive Decay}

\section{Transmutation and Nuclear Energy}

\section{Uses of Radioisotopes}

\section{Biological Effects of Radiation}

\backmatter
\chapter{Errata}

\end{document}
