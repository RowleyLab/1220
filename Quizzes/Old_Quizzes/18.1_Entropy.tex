\documentclass[11pt, letterpaper]{memoir}
\usepackage{HomeworkStyle}
\geometry{margin=0.75in}

\begin{document}
	\begin{center}
		{\large	Quiz 18.1 -- Entropy}
	\end{center}
	{\large Name: \rule[-1mm]{4in}{.1pt} 

\subsection*{Question 1}
For each process, indicate whether the entropy change for the system is positive or negative:
\begin{itemize}
	\item \ch{H2O(g) -> H2O(l)}
	\item \ch{CO2(s) -> CO2(g)}
	\item \ch{N2(g) + 3 H2(g) -> 2 NH3(g)}
	\item \ch{Na2CO3(aq) + H2SO4(aq) -> Na2SO4(aq) + 2 H2O(l) + CO2(g)}
	\item \ch{NaC2H3O2(aq) -> NaC2H3O2(s)}
	\item The valve on a pressurized gas tank is opened, and gas comes streaming out
	\item A sample is taken out of the freezer and left on the counter for twenty minutes
\end{itemize}

\subsection*{Question 2}
Use the Boltzmann definition of entropy to find the entropy of each system:
\begin{itemize}
	\item Tossing a single six-sided die
	
	\vspace{3em}
	\item Tossing two six-sided dice (consider the dice to be distinguishable)
	
	\vspace{3em}
	\item A $0.0100~g$ diamond is cooled to near absolute zero. The only excited state accessible at this temperature is a nuclear spin state, which each carbon atom having an equal probability of being in the excited or ground state (consider the carbon atoms to be distinguishable)
	
	\vspace{3em}
	\item $4$ \ch{He} atoms are placed in a chamber with two halves. Each atom has equal probability to be in either half of the chamber (consider the \ch{He} atoms to be indistinguishable)
	
	\vspace{3em}
	\item $4$ different gas particles are placed in a chamber with two halves. Each atom has equal probability to be in either half of the chamber (consider the particles to be distinguishable)
\end{itemize}
	\newpage
	\newgeometry{margin=1.25in}
	\pagestyle{empty}
	\addtocounter{page}{-1}
\section*{\emph{{\fontspec{Malgun Gothic}오늘} (Today)}}
\paragraph{By {\fontspec{Malgun Gothic}구상} (Ku Sang)}~

{\fontspec{Malgun Gothic}
	\begin{verse}
		오늘도 신비의 샘인 하루를 맞는다.
		
		이 하루는 저 강물의 한 방울이\\
		어느 산골짝 옹달샘에 이어져 있고\\
		아득한 푸른 바다에 이어져 있듯\\
		과거와 미래와 현재가 하나다.
		
		이렇듯 나의 오늘은 영원 속에 이어져\\
		바로 시방 나는 그 영원을 살고 있다.
		
		그래서 나는 죽고 나서부터가 아니라\\
		오늘서부터 영원을 살아야 하고\\
		영원에 합당한 삶을 살아야 한다.
		
		마음이 가난한 삶을 살아야 한다.\\
		마음을 비운 삶을 살아야 한다.
	\end{verse}
}

\vspace{2em}
\begin{verse}
	Today again I meet a day, a well of mystery.
	
	Like a drop of that river extends to\\
	a spring of a valley and then to\\
	the faraway blue sea, for this day\\
	the past, the future, and the present are one.
	
	So does my today extend to eternity,\\
	and right now I am living the eternity.
	
	So, starting from today, I should live\\
	eternity, not after I die,\\
	and should live a life that deserves eternity.
	
	I should live the life of a poor heart.\\
	I should live the life of an empty heart.
\end{verse}
\end{document}
