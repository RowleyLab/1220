\documentclass[12pt, letterpaper]{memoir}
\usepackage{HomeworkStyle}

\begin{document}
	\begin{center}
		{\large	Exam 1 Feedback Quiz}
	\end{center}
	{\large Name: \rule[-1mm]{4in}{.1pt} {\bfseries (10 points)}\hspace{4em}$\dfrac{~}{10}$} 
	\subsection*{Question 1 (0 points)}
	An examine must be able to distinguish between students who deserve an ``A,'' ``B,'' ``C,'' etc. Exams which are too difficult \emph{or} too easy fail at that purpose. How do you feel like this exam was:
	
	{\large Too Easy} \hspace{2em} {\large Just Right} \hspace{2em} {\large Too Difficult}

	\vspace{2em}
	\subsection*{Question 2 (0 points)}
	Exams which are too long (too many questions) place more emphasis on strategic test-taking rather than on subject mastery, and give a disadvantage to students with anxiety, reading issues, etc. Exams which are too short give fewer opportunities to show broad knowledge, and make any small mistake have a large impact on the final score.  How do you feel like this exam was:
	
	{\large Too Short} \hspace{2em} {\large Just Right} \hspace{2em} {\large Too Long}
	
	\vspace{2em}
	\subsection*{Question 3 (0 points)}
	The point of switching to a digital textbook was for your convenience, so you could have it available wherever you have an Internet connection. The peril is that you can more easily ignore a textbook which doesn't take up physical space in your life. How much are you \emph{actually} reading the textbook?
	
	\noindent(I realize this quiz isn't technically anonymous, but I won't use your answers here against you. This is to help me deliver a better course, so please be honest)
	
	{\large Every Section} \hspace{2em} {\large Only Some Sections} \hspace{2em} {\large Only as a Reference} \hspace{2em} {\large Hardly at All}
	
	\vspace{2em}
	\subsection*{Question 4 (0 points)}
	Please share any comments you have about the first exam, or about the course generally.
\end{document}
