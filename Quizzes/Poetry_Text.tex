\documentclass[12pt, openany, letterpaper]{memoir}
\usepackage{MyStyle}
\pagestyle{empty}
\newgeometry{hmargin=0.85in}

\begin{document}

\mainmatter

\newpage
\newpage
\section*{\emph{[i carry your heart with me(i carry it in]}}
\paragraph{By ee cummings}~
\begin{verse}
	i carry your heart with me(i carry it in\\
	my heart)i am never without it(anywhere\\
	i go you go,my dear;and whatever is done\\
	by only me is your doing,my darling)\\
	\hspace{14em} i fear\\
	no fate(for you are my fate,my sweet)i want\\
	no world(for beautiful you are my world,my true)\\
	and it’s you are whatever a moon has always meant\\
	and whatever a sun will always sing is you
	
	here is the deepest secret nobody knows\\
	(here is the root of the root and the bud of the bud\\
	and the sky of the sky of a tree called life;which grows\\
	higher than soul can hope or mind can hide)\\
	and this is the wonder that's keeping the stars apart
	
	i carry your heart(i carry it in my heart)
\end{verse}


\newpage
\section*{\emph{A Poem for Pulse}}
\paragraph{By Jameson Fitzpatrick}~
\begin{verse}
	Last night, I went to a gay bar\\
	with a man I love a little.\\
	After dinner, we had a drink.\\
	We sat in the far-back of the big backyard\\
	and he asked, What will we do when this place closes?\\
	I don't think it's going anywhere any time soon, I said,\\
	though the crowd was slow for a Saturday,\\
	and he said—Yes, but one day. Where will we go?\\
	He walked me the half-block home\\
	and kissed me goodnight on my stoop—\\
	properly: not too quick, close enough\\
	our stomachs pressed together\\
	in a second sort of kiss.\\
	I live next to a bar that's not a gay bar\\
	—we just call those bars, I guess—\\
	and because it is popular\\
	and because I live on a busy street,\\
	there are always people who aren't queer people\\
	on the sidewalk on weekend nights.\\
	Just people, I guess.\\
	They were there last night.\\
	As I kissed this man I was aware of them watching\\
	and of myself wondering whether or not they were just.\\
	But I didn't let myself feel scared, I kissed him\\
	exactly as I wanted to, as I would have without an audience,\\
	because I decided many years ago to refuse this fear—\\
	an act of resistance. I left\\
	the idea of hate out on the stoop and went inside,\\
	to sleep, early and drunk and happy.\\
	While I slept, a man went to a gay club\\
	with two guns and killed forty-nine people.\\
	Today in an interview, his father said he had been disturbed\\
	recently by the sight of two men kissing.\\
	What a strange power to be cursed with:\\
	for the proof of men's desire to move men to violence.\\
	What's a single kiss? I've had kisses\\
	no one has ever known about, so many\\
	kisses without consequence—\\
	but there is a place you can't outrun,\\
	whoever you are.\\
	There will be a time when.\\
	It might be a bullet, suddenly.\\
	The sound of it. Many.\\
	One man, two guns, fifty dead—\\
	Two men kissing. Last night\\
	I can't get away from, imagining it, them,\\
	the people there to dance and laugh and drink,\\
	who didn't believe they'd die, who couldn't have.\\
	How else can you have a good time?\\
	How else can you live?\\
	There must have been two men kissing\\
	for the first time last night, and for the last,\\
	and two women, too, and two people who were neither.\\
	Brown people, which cannot be a coincidence in this country\\
	which is a racist country, which is gun country.\\
	Today I'm thinking of the Bernie Boston photograph\\
	Flower Power, of the Vietnam protestor placing carnations\\
	in the rifles of the National Guard,\\
	and wishing for a gesture as queer and simple.\\
	The protester in the photo was gay, you know,\\
	he went by Hibiscus and died of AIDS,\\
	which I am also thinking about today because\\
	(the government's response to) AIDS was a hate crime.\\
	Now we have a president who names us,\\
	the big and imperfectly lettered us, and here we are\\
	getting kissed on stoops, getting married some of us,\\
	some of us getting killed.\\
	We must love one another whether or not we die.\\
	Love can't block a bullet\\
	but neither can it be shot down,\\
	and love is, for the most part, what makes us—\\
	in Orlando and in Brooklyn and in Kabul.\\
	We will be everywhere, always;\\
	there's nowhere else for us, or you, to go.\\
	Anywhere you run in this world, love will be there to greet you.\\
	Around any corner, there might be two men. Kissing.
\end{verse}

\newpage
\section*{\emph{The Journey}}
\paragraph{By Mary Oliver}~
\begin{verse}
	One day you finally knew\\
	what you had to do, and began,\\
	though the voices around you\\
	kept shouting\\
	their bad advice –\\
	though the whole house\\
	began to tremble\\
	and you felt the old tug\\
	at your ankles.\\
	“Mend my life!”\\
	each voice cried.\\
	But you didn’t stop.\\
	You knew what you had to do,\\
	though the wind pried\\
	with its stiff fingers\\
	at the very foundations,\\
	though their melancholy\\
	was terrible.\\
	It was already late\\
	enough, and a wild night,\\
	and the road full of fallen\\
	branches and stones.\\
	But little by little,\\
	as you left their voices behind,\\
	the stars began to burn\\
	through the sheets of clouds,\\
	and there was a new voice\\
	which you slowly\\
	recognized as your own,\\
	that kept you company\\
	as you strode deeper and deeper\\
	into the world,\\
	determined to do\\
	the only thing you could do –\\
	determined to save\\
	the only life you could save.
\end{verse}

\newpage
\newgeometry{hmargin=1.25in,vmargin=1.2in}
\section*{\emph{When Death Comes}}
\paragraph{By Mary Oliver}~
\begin{verse}
	When death comes\\
	like the hungry bear in autumn;\\
	when death comes and takes all the bright coins from his purse
	
	to buy me, and snaps the purse shut;\\
	when death comes\\
	like the measle-pox
	
	when death comes\\
	like an iceberg between the shoulder blades,
	
	I want to step through the door full of curiosity, wondering:\\
	what is it going to be like, that cottage of darkness?
	
	And therefore I look upon everything\\
	as a brotherhood and a sisterhood,\\
	and I look upon time as no more than an idea,\\
	and I consider eternity as another possibility,
	
	and I think of each life as a flower, as common\\
	as a field daisy, and as singular,
	
	and each name a comfortable music in the mouth,\\
	tending, as all music does, toward silence,
	
	and each body a lion of courage, and something\\
	precious to the earth.
	
	When it’s over, I want to say all my life\\
	I was a bride married to amazement.\\
	I was the bridegroom, taking the world into my arms.
	
	When it’s over, I don’t want to wonder\\
	if I have made of my life something particular, and real.
	
	I don’t want to find myself sighing and frightened,\\
	or full of argument.
	
	I don’t want to end up simply having visited this world.
\end{verse}

\newpage
\section*{\emph{Kindness}}
\paragraph{By Naomi Shihab Nye}~
\begin{verse}
	Before you know what kindness really is\\
	you must lose things,\\
	feel the future dissolve in a moment\\
	like salt in a weakened broth.\\
	What you held in your hand,\\
	what you counted and carefully saved,\\
	all this must go so you know\\
	how desolate the landscape can be\\
	between the regions of kindness.\\
	How you ride and ride\\
	thinking the bus will never stop,\\
	the passengers eating maize and chicken\\
	will stare out the window forever.
	
	Before you learn the tender gravity of kindness\\
	you must travel where the Indian in a white poncho\\
	lies dead by the side of the road.\\
	You must see how this could be you,\\
	how he too was someone\\
	who journeyed through the night with plans\\
	and the simple breath that kept him alive.
	
	Before you know kindness as the deepest thing inside,\\
	you must know sorrow as the other deepest thing.\\
	You must wake up with sorrow.\\
	You must speak to it till your voice\\
	catches the thread of all sorrows\\
	and you see the size of the cloth.\\
	Then it is only kindness that makes sense anymore,\\
	only kindness that ties your shoes\\
	and sends you out into the day to gaze at bread,\\
	only kindness that raises its head\\
	from the crowd of the world to say\\
	It is I you have been looking for,\\
	and then goes with you everywhere\\
	like a shadow or a friend.	
\end{verse}

\newpage
\newgeometry{hmargin=1.25in,vmargin=0.9in}
\section*{\emph{A Brave and Startling Truth}}
\paragraph{By Maya Angelou}~
\begin{verse}
	We, this people, on a small and lonely planet\\
	Traveling through casual space\\
	Past aloof stars, across the way of indifferent suns\\
	To a destination where all signs tell us\\
	It is possible and imperative that we learn\\
	A brave and startling truth
	
	And when we come to it\\
	To the day of peacemaking\\
	When we release our fingers\\
	From fists of hostility\\
	And allow the pure air to cool our palms
	
	When we come to it\\
	When the curtain falls on the minstrel show of hate\\
	And faces sooted with scorn are scrubbed clean\\
	When battlefields and coliseum\\
	No longer rake our unique and particular sons and daughters\\
	Up with the bruised and bloody grass\\
	To lie in identical plots in foreign soil
	
	When the rapacious storming of the churches\\
	The screaming racket in the temples have ceased\\
	When the pennants are waving gaily\\
	When the banners of the world tremble\\
	Stoutly in the good, clean breeze
	
	When we come to it\\
	When we let the rifles fall from our shoulders\\
	And children dress their dolls in flags of truce\\
	When land mines of death have been removed\\
	And the aged can walk into evenings of peace\\
	When religious ritual is not perfumed\\
	By the incense of burning flesh\\
	And childhood dreams are not kicked awake\\
	By nightmares of abuse
	
	When we come to it\\
	Then we will confess that not the Pyramids\\
	With their stones set in mysterious perfection\\
	Nor the Gardens of Babylon\\
	Hanging as eternal beauty\\
	In our collective memory\\
	Not the Grand Canyon\\
	Kindled into delicious color\\
	By Western sunsets
	
	Nor the Danube, flowing its blue soul into Europe\\
	Not the sacred peak of Mount Fuji\\
	Stretching to the Rising Sun\\
	Neither Father Amazon nor Mother Mississippi who, without favor,\\
	Nurture all creatures in the depths and on the shores\\
	These are not the only wonders of the world
	
	When we come to it\\
	We, this people, on this minuscule and kithless globe\\
	Who reach daily for the bomb, the blade and the dagger\\
	Yet who petition in the dark for tokens of peace\\
	We, this people on this mote of matter\\
	In whose mouths abide cankerous words\\
	Which challenge our very existence\\
	Yet out of those same mouths\\
	Come songs of such exquisite sweetness\\
	That the heart falters in its labor\\
	And the body is quieted into awe
	
	We, this people, on this small and drifting planet\\
	Whose hands can strike with such abandon\\
	That in a twinkling, life is sapped from the living\\
	Yet those same hands can touch with such healing, irresistible tenderness\\
	That the haughty neck is happy to bow\\
	And the proud back is glad to bend\\
	Out of such chaos, of such contradiction\\
	We learn that we are neither devils nor divines
	
	When we come to it\\
	We, this people, on this wayward, floating body\\
	Created on this earth, of this earth\\
	Have the power to fashion for this earth\\
	A climate where every man and every woman\\
	Can live freely without sanctimonious piety\\
	Without crippling fear
	
	When we come to it\\
	We must confess that we are the possible\\
	We are the miraculous, the true wonder of this world\\
	That is when, and only when\\
	We come to it.
\end{verse}

\end{document}
